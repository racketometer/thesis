\chapter{Deployment}
This chapter describes how the applications are distributed and deployed and how the production systems are setup.

\section{Deployment model}
The software packages are deployed as a distributed system as shown on figure \ref{fig:deploymentModel}.
The front-end codebase is compiled to native applications on the targeted platforms and deployed to the devices.
The back-end codebase is deployed to a \glsi{heroku} infrastructure running \glsi{node} and is setup to communicate with a \glsi{mongodb} instance hosted on \glsi{mlab}, see section \ref{sec:staging}.

\graphic{1}{deployment}{Software deployment model}{fig:deploymentModel}

\section{Back-end}
The back-end application is deployed on \glsi{heroku} with free hosting. 
This is because the product is still in a beta state and the cheapest solution, i.e free, is chosen.

When deploying a \glsi{node} application on \glsi{heroku}, you deploy your source code from a local git repository.
\Glsi{heroku} then builds the application with \glsi{npm} scripts and starts the application on the running web instance as illustrated on \ref{fig:deploymentFlow}.

\graphic{0.8}{deploymentFlow}{\Glsi{heroku} deployment flow}{fig:deploymentFlow}

\subsection{Existing server}
To deploy to the existing server push git changes to the \glsi{heroku} remote with the following command.

\verb+git push heroku master+

\subsection{New server}
\label{sec:deployment:new}
To deploy the application to a new server follow the \glsi{heroku} guide for deploying a \glsi{node} application on their servers which breaks down the different commands \citep{documentation:deployment:heroku}.

When the application is deployed, the \verb+npm install+ command is automatically run to install the dependencies.
To make sure that \glsi{npm} installs the developer dependencies, the \verb+NPM_CONFIG_PRODUCTION+ environment variable is set to \verb+false+ with the following command in the \glsi{heroku} \gls{cli}.

\verb+heroku config:set NPM_CONFIG_PRODUCITON=false+

\glsi{heroku} is configured with a \verb+Procfile+ \citep{documentation:deployment:heroku:procfile}, with the single command below. This defines how the web application is started.

\verb+web: node dist/server.js+

If no \verb+Procfile+ is provided, \glsi{heroku} will run the \verb+npm start+ command.
That command is used under development and runs a development server.
Therefore a \verb+Procfile+ is provided to run a different command.

To make sure that the source code gets compiled before the server starts, a \verb+postinstall+ script is added to the \verb+package.json+ file.
This script will run the \glsi{webpack} build chain that compiles the \gls{typescript} code and outputs it to the \verb+dist+ folder.

\subsection{Staging}
\label{sec:staging}
As the application uses staging for different environments, several environment variables need to be set.
Specifically the \verb+NODE_ENV+ and the \verb+ROM_DB_PROD+ variable. 
The \verb+NODE_ENV+ variable defines if a database seed should be applied on startup or not. 
This is used on the testing stage so that integration tests are always run against the same data.

In production we do not ever want to reseed our database and therefore the \verb+NODE_ENV+ variable is set to \verb+prod+.

Lastly \verb+ROM_DB_PROD+ needs to be set to the \gls{url} of the production \glsi{mongodb} instance.
For a complete reference on environment variables see chapter \ref{ch:environmentVariables}.

The \glsi{mongodb} for production is hosted on \glsi{mlab} as Database-as-a-Service web service.
