\chapter{Deployment}
This chapter describes the how the applications are distributed and deployed.

\section{Back-end}
The back-end application is deployed on \glsi{heroku} with free hosting. This is because the product is still in a beta state and the cheapest solution, i.e free, is chosen.

\Glsi{heroku} have a guide for deploying a \glsi{node} application on their servers which breaks down the different commands \citep{documentation:deployment:heroku}.

When deploying a \glsi{node} application on \glsi{heroku}, you deploy your \glsi{github} repository. 
This means that you are not deploying your compiled code, but rather your source code.
This has the side effect that you need your build chain installed on the \glsi{node} server.
To make sure that \glsi{npm} installs your developer dependencies, the \verb+NPM_CONFIG_PRODUCTION+ environment variable needs to be set to \verb+false+.
This is done with the following command in the \gls{cli}

\verb+heroku config:set NPM_CONFIG_PRODUCITON=false+

\Glsi{heroku} allows for the use of a \verb+Procfile+ to specify commands for different types of applications \citep{documentation:deployment:heroku:procfile}.
If no \verb+Procfile+ is provided, \glsi{heroku} will run the \verb+npm start+ command.
That command is used under development and runs a development server.
Therefore a \verb+Procfile+ is provided to run a different command than \verb+npm start+.

When the application is deployed, the \verb+npm install+ command is automatically run to install the dependencies. 
To make sure that the source code gets compiled before the server starts, a \verb+postinstall+ script is added to the \verb+package.json+ file.

The \verb+postinstall+ script will run the build chain through \verb+webpack+. 
This compiles the \gls{typescript} code and outputs it in the \verb+dist+ folder.

\Glsi{heroku} then looks at the \verb+Procfile+ for a web command, since this is a web application and runs this command:

\verb+web: node dist/server.js+

This starts the \glsi{node} server with \verb+dist/server.js+ as entry point.

\subsection{Staging}
As the application uses staging for different environments, several environment variables need to be set.
Specifically the \verb+NODE_ENV+ and the \verb+ROM_DB_PROD+ variable. 
The \verb+NODE_ENV+ variable defines if a database seed should be applied on startup or not. 
This is used on the testing stage so that integration tests are always run against the same data.

In production we do not ever want to reseed our database and therefore the \verb+NODE_ENV+ variable is set to \verb+prod+.

Lastly \verb+ROM_DB_PROD+ needs to be set to the production \glsi{mongodb} instance.