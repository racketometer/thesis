\chapter{Architecture}
This chapter describes the software architectures for the front and back-end applications.
It also covers what major frameworks are used and to what purpose.

\section{Frameworks}
The system is build with different frameworks as seen on figure  \ref{fig:frameworkArchitecture}.
The front-end application uses \glsi{ng2} as the main framework.
It utilizes \glsi{nativescript} as the \gls{ui} rendering layer and \glsi{graphql} as the back-end communication specification and \glsi{apollo:client} for request execution.

The back-end uses the \glsi{apollo:server} framework to manage communications with the \glsi{graphql} specification and uses \glsi{mongodb} for data persistence.

\graphic{1}{frameworkArchitecture}{System framework architecture}{fig:frameworkArchitecture}

\section{Software architecture}
Each application uses different architectures and the source code is structured differently.

\subsection{Front-end}
The front-end application is based on a component structured \gls{mvc} architecture, that is standard for \glsi{ng2} applications.
Each component consists of several files responsible for different parts of the component as shown on figure \ref{fig:componentFiles}.
Note that the \verb+html+ files do not comply with the \gls{html} specification.
Instead it is an enhanced \gls{xml} notation with \glsi{nativescript} and custom components.

\graphic{0.45}{componentFiles}{Source files for a component}{fig:componentFiles}

The naming convention used is as follows with the exception of common style sheets.

\verb+<name>.<type>.<extension>+

In \glsi{ng2} component declarations, listing \ref{lst:component}, file names are omitting platform names as seen on line 4.
This is due to the \glsi{nativescript} tool-chain.
If a file is not found, it adds the targeted platform to the last part of the file name to match platform specific versions.
The purpose of this is to allow target specific details, e.g. styling.

\lstsetjavascript
\begin{lstlisting}[caption=Login component declaration, label=lst:component]
@Component({
  selector: "rom-login",
  templateUrl: "login/login.component.html",
  styleUrls: ["login/login-common.css", "login/login.component.css"],
})
export class LoginComponent implements OnInit {
  ...
}
\end{lstlisting}

\subsection{Back-end}
The back-end application is based on a three layer architecture, as seen on figure \ref{fig:backendArchitecture}.
The \textit{Schema} layer handles \gls{api} exposure to clients and is written in \glsi{graphql} schema language.
The \textit{Resolvers} layer handles field resolvement for queries.
This is where decisions on where data is to be fetched from is handled.
The bottom layer is a service layer including data access and utility services.
This includes database connections and other data access logic as well as e-mail services and data analysis algorithms.

\graphic{1}{backend_architecture}{Back-end architecture}{fig:backendArchitecture}
