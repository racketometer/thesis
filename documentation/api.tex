\chapter{API documentation}
This chapter will cover the documentation of the back-end \gls{api}.

For reference of the \glsi{graphql} documentation \citep{graphql:typesystem} the following notations are used through out the documentation.

\begin{itemize}  
\item \verb+!+ Parameter is required
\item \verb+[SomeType]+ Array of \verb+SomeType+
\end{itemize}

Here follows descriptions of each resource found on the \gls{api} endpoint.

\section{Root}
On figure \ref{fig:root} the two root types are shown. 
These are \verb+Query+ and \verb+Mutation+. 
\verb+Query+ can be compared to \gls{http} GET requests and \verb+Mutation+s can be considered as \gls{http} POST requests.
\graphich{0.6}{api/rootType}{Root types}{fig:root}

\section{Query}
On figure \ref{fig:query} the \verb+Query+ root type is shown. It contains two fields. 

The first field \verb+viewer+ is where the restricted data is located.
It takes a \verb+token+ parameter, and returns a \verb+viewer+ type which is shown on figure \ref{fig:viewerType}.

The second field \verb+login+ also returns a \verb+viewer+ type. 
This is to save a round trip.
The viewer type has a \verb+token+ parameter that can be used to query the \verb+viewer+ field at a later time after logging in.
\graphich{0.6}{api/queryType}{The Query root type with fields}{fig:query}

\section{Viewer}
The \verb+Viewer+ can only be retrieved if the user is authenticated. 
Either through the \verb+viewer+ field or \verb+login+ field on the root type \verb+query+. The \verb+Viewer+ is shown on figure \ref{fig:viewerType}
\graphich{0.6}{api/viewerType}{The Query root type with fields}{fig:viewerType}

\section{Mutation}
The \verb+Mutation+ root type is also a restricted field where the user needs to be authenticated. 
It takes a \verb+token+ parameter which is required. 
It returns a \verb+MutationViewer+ that is shown on figure \ref{fig:mutationViewer}
\graphich{0.6}{api/mutationType}{The Mutation root type with fields}{fig:mutationType}

\section{MutationViewer}
The \verb+MutationViewer+ has 3 functions that takes arguments and a \verb+logout+ field that just needs to be required on the \verb+MutationViewer+ to be called. The \verb+MutationViewer+ can be seen on figure \ref{fig:mutationViewer}
\graphich{0.6}{api/mutationViewerType}{The MutationViewer type with fields}{fig:mutationViewer}

\section{CreateUser}
The \verb+createUser+ function is used to create users with minimal information. 
As shown on figure \ref{fig:createUser}, it only takes an \verb+email+ and a \verb+password+.
\graphich{0.6}{api/createUserType}{The createUser function with arguments}{fig:createUser}

\section{CreateAutoUser}
The \verb+createAutoUser+ function can be used by consultants and coaches to create a user with an autogenerated password.
The user then receives the password on the provided email.
It takes an \verb+AutoUser+ as an input parameter.
\graphich{0.6}{api/createAutoUserType}{The createAutoUser function with arguments}{fig:createAutoUser}

\section{AutoUser}
\verb+AutoUser+ is an input type used by \verb+createAutoUser+ and can be seen on figure \ref{fig:autoUser}.
\graphich{0.6}{api/autoUserInputType}{The AutoUser input type}{fig:autoUser}

\section{ChangePassword}
\verb+changePassword+ can be used to change password at any time, as long as the old password is provided.
The \verb+changePassword+ function can be seen on figure \ref{fig:changePassword}
\graphich{0.6}{api/changePasswordType}{The changePassword function with arguments}{fig:changePassword}

\section{Interactive API exploration}
When using \glsi{graphql}, \glsi{graphiql} is included.
\Glsi{graphiql} is a live web documentation of the schema that can be used to browse the documentation and also query against the application.
This is useful for both for the back-end developer that does not need to maintain an \gls{api} documentation and also for the front-end developer that can browse the schema live instead of referring to a hard copy, that might be out dated.

An example of how you can use \glsi{graphiql} is shown on figure \ref{fig:graphiql}.
The view is split into three columns.
The column to the far left is where you can write queries.
The center column is for the \glsi{graphql} responses.
To the right is the documentation explorer.

The example shows a login query using \verb+email+ and \verb+password+ variables.
When pushing the \textit{Play} button, the query is run and the response is shown with exactly the two parameters asked for, i.e. \verb+_id+ and \verb+token+.

\Glsi{graphiql} for the running back-end is available on the following \gls{url}: \newline
\url{https://polar-reef-40902.herokuapp.com/graphiql}

Follow this \gls{url} for a prepared query setup: \newline
\url{https://goo.gl/pQjOko}

\graphic{1}{api/graphiql}{\Glsi{graphiql} web interface}{fig:graphiql}
