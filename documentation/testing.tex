\chapter{Testing}
This chapter describes how testing is set up and managed a cross the applications.

\section{Tools}
The back-end and front-end both utilize the same tools for testing.
\glsi{mocha} is the test-runner, \glsi{sinon} is used for mocking dependencies and \glsi{chai} is used as the assertion library.

\glsi{istanbul} calculates code coverage and \glsi{coveralls} posts it to the \glsi{coveralls} web service and adds the information to \glsi{github} in form of comments on pull-requests and as a badge on the master branch.
On pull-requests you also get information about coverage increase and decrease.

Tests are also run on \glsi{travis}, see section \ref{sec:ci}.

\section{Running tests}
The front-end and back-end uses the same commands for running tests and posting code coverage.
To run tests on the local development machine, environment variables need to be set before running the below command in the root of the specific projects folder, see chapter \ref{ch:environmentVariables}.

The following command will run all tests against the test database, which is reseeded on each run to ensure the same data is available.
Code coverage results are output to the \verb+coverage+ folder.

\verb+npm test+

To post the calculated code coverage to \glsi{coveralls} use the below command.

\verb+npm run coverage+

\section{Continuous integration}
\label{sec:ci}
Both repositories are connected to a \glsi{ci} server, \glsi{travis}.
The tasks done by the server is configured in the \verb+.travis.yml+ file in the root of the repositories.

For both repositories the source code is linted, build and tested.

On the front-end, the \textit{Android} \glspl{sdk} are installed prior to running the afore mentioned tasks.

On each commit to the repositories or when a pull-request is opened, a build is initiated.
If the build fails, an email is send to the author of the commit.
In the case of pull-requests, \glsi{travis} leaves an indication on \glsi{github} on the build status to make it easier for the reviewer.

