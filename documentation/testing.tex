\chapter{Testing}
This chapter describes how testing is set up and managed a cross the applications.

\section{Tools}
The back-end and front-end both utilize the same tools for testing.
\glsi{mocha} is the test-runner used for running the tests.
\glsi{sinon} is used for spys, mocks and stubs. \glsi{chai} is used as the assertion library.

\glsi{istanbul} calculates coverage and \glsi{coveralls} posts it to their web service and adds the information to \glsi{github} in form of comments on pull-requests and as a badge for the master branch coverage.
On pull-requests you also get information about coverage increase and decrease.

Tests are also run on \glsi{travis}, see chapter \fxfatal{add ref to CI chapter}

\section{Running tests}
The front-end and back-end uses the same commands for running tests and posting coverage.
To run tests on the local development machine, environment variables need to be set before running the below command in the root of the specific projects folder, see section \fxfatal{add environment variable section ref}.

The following command will run all tests against the test database.
The database is reseeded each time to ensure the same data is available. 
Coverage results are output to the \verb+coverage+ folder.

\verb+npm run test+

To post the calculated code coverage to \glsi{coveralls} use the below command.

\verb+npm run coverage+
