\chapter{Acceptance tests protocol}
\label{ch:protocol}
This appendix describes the results and experiences during the completion of the acceptance test.

Present: Bjørn Sørensen and Jesper Christensen

Date: 14/12/2016

Three user accounts are used in the tests.

\begin{enumerate}
\item Athlete - Email: johnny@test.dk, password: 1234
\item Coach - Email: coach@test.dk, password: 1234
\item Consultant - Email: consultant@test.dk, password: 1234
\end{enumerate}

\userstory{As a user, I want to use a phone or tablet to access the system}

Device operating systems to support includes: \newline
\begin{tabularx}{\textwidth}{X}
    \textit{Apple iOS} 9 and 10 \\
    \textit{Android} 4.3, 4.4, 5.0, 5.1, 6.0 and 7.0
\end{tabularx}

To limit the scope of the project, only \textit{Android} version 5.0 is tested on hardware and version 6.0 on an \textit{Android} emulator.

\begin{tabularx}{\textwidth}{|b|c|c|}
	\hline
	 & Yes & No \\
	\hline
	As a user, I can download the app from \textit{iOS}'s app store \textit{AppStore} using a phone or tablet with either one of the supporting operating systems &   & \xmark  \\
	\hline
	As a user, I can download the app from \textit{Android}'s app store \textit{Google Play} using a phone or tablet with either one of the supporting operation systems & \xmark  &  \\
	\hline
	As a user, I can log in and access the system in the downloaded application on \textit{iOS} &  & \xmark \\
		\hline
	As a user, I can log in and access the system in the downloaded application on \textit{Android} & \xmark  &  \\
	\hline
\end{tabularx}
	
\vspace{3mm}
\textbf{Comments:}
The application was available on the \textit{Android} app store. 
It was downloaded and login was successful using both the all three user types, on a LenovoPad TB3 running \textit{Android} version 5.0.1.

\vspace{3mm}

\userstory{As a consultant / coach, I want to input user information when beginning a session and automatically add them to the user database}

If Internet is present, the user information must be submitted immediately to the database.

If Internet is not present, the user information must be stored on the device for later automatic submission.

\begin{tabularx}{\textwidth}{|b|c|c|}
	\hline
	 & Yes & No \\
	\hline
	As a consultant, I can input user information on their behalf & \xmark  &  \\
	\hline
	As a coach, I can input user information on their behalf & \xmark  &  \\
	\hline
	If Internet is present, the user will be added to the database immediately & \xmark & \\
	\hline
	If Internet is unavailable, the user will be stored on the device and added to the database when internet connection is restored &  & \xmark  \\
	\hline
\end{tabularx}
	
\vspace{3mm}
\textbf{Comments:}
When logged in as the coach, user information, excluding the \verb+isCoach+ flag, could be input on their behalf.
When logged in as the consultant, user information could be input on their behalf.
When submitting the form, the user can be found in the Users tab.
\vspace{3mm}

\userstory{As a consultant, I want to start a session with an athlete and record data from the racket}

\begin{tabularx}{\textwidth}{|b|c|c|}
	\hline    
	 & Yes & No \\
	\hline
	As a consultant, I can start a session with a user and record data from the racket &   & \xmark  \\
	\hline
\end{tabularx}
	
\vspace{3mm}
\textbf{Comments:}
As no racket sensor was available, this test could not be performed.
\vspace{3mm}

\userstory{As a consultant, I want started sessions to be persisted between application restarts}

\begin{tabularx}{\textwidth}{|b|c|c|}
	\hline
	 & Yes & No \\
	\hline
	As a consultant, I can restart the application and still have the same started sessions as before the restart &   & \xmark  \\
	\hline
\end{tabularx}

\vspace{3mm}
\textbf{Comments:}
As no racket sensor was available, this test could not be performed. 
The functionality is present in code however.
\vspace{3mm}

\userstory{As a consultant / coach, I can find a previously stored user}

\begin{tabularx}{\textwidth}{|b|c|c|}
	\hline
	 & Yes & No \\
	\hline
	As a consultant / coach, I can find a previously stored user & \xmark  &  \\
	\hline
\end{tabularx}

\vspace{3mm}
\textbf{Comments:}
In the Users tab all users created by the consultant can be found.
\vspace{3mm}

\userstory{As a consultant, I want to show details of recorded data to specific users}

\begin{tabularx}{\textwidth}{|b|c|c|}
	\hline
	 & Yes & No \\
	\hline
	As a consultant, I can show details of recorded data &   & \xmark  \\
	\hline
\end{tabularx}

\vspace{3mm}
\textbf{Comments:}
As no racket sensor was available, this test could not be performed. 
The functionality is present in code however.
\vspace{3mm}

\userstory{As a user, I want to receive an email with a generated password for my account, when a consultant / coach have created an account on my behalf}

\begin{tabularx}{\textwidth}{|b|c|c|}
	\hline
	 & Yes & No \\
	\hline
	As a user, I receive an email with a generated password to my account, when a consultant have created one on my behalf & \xmark & \\
	\hline
	As a user, I receive an email with a generated password to my account, when a coach have created one on my behalf & \xmark & \\
	\hline
\end{tabularx}

\vspace{3mm}
\textbf{Comments:}
An email was received with a password.
\vspace{3mm}

\userstory{As a user, I want to be able to change my auto-genereated password from within the application}

If internet is not present, this operation is not possible.

\begin{tabularx}{\textwidth}{|b|c|c|}
	\hline
	 & Yes & No \\
	\hline
	As a user, I can change my auto-generated password from within the application, if I have internet access & \xmark & \\
	\hline
\end{tabularx}

\vspace{3mm}
\textbf{Comments:}
When logging on with a new user, they are forced to change their auto-generated password.
\vspace{3mm}

\userstory{As an athlete, I want to compare my performance data to other athletes}

\begin{tabularx}{\textwidth}{|b|c|c|}
	\hline
	 & Yes & No \\
	\hline
	As an athlete, I can compare my performance data with other athletes &   & \xmark  \\
	\hline
\end{tabularx}

\vspace{3mm}
\textbf{Comments:}
This feature did not make it into the prototype.
\vspace{3mm}

\userstory{As a user, I want to share my user profile on social media}

\begin{tabularx}{\textwidth}{|b|c|c|}
	\hline
	 & Yes & No \\
	\hline
	As a user, I can share my user profile on Facebook &   & \xmark  \\
	\hline
	As a user, I can share my user profile on Twitter &   & \xmark  \\
	\hline
\end{tabularx}

\vspace{3mm}
\textbf{Comments:}
This feature did not make it into the prototype.
\vspace{3mm}

\userstory{As an athlete, I want to see my recorded data compared to historical data}

\begin{tabularx}{\textwidth}{|b|c|c|}
	\hline
	 & Yes & No \\
	\hline
	As an athlete, I can see my recorded data compared to historical data if I have multiple recording sessions &   & \xmark  \\
	\hline
\end{tabularx}

\vspace{3mm}
\textbf{Comments:}
This feature did not make it into the prototype.
\vspace{3mm}
