\chapter{Consultant interview}
\label{ch:consultantsInterview}
This is a transcription of a semi structured interview with two representative consultants.
The interview was conducted in danish as both consultants are danish citizens.

\textbf{Interviewer}: \newline
Bjørn Sørensen \newline
Jesper O. Christensen

\textbf{Interviewees}: \newline 
Mark Boesen, marcboesen@hotmail.com \newline
Kaare Moss, kaaremoss@gmail.com

\textbf{Date}: 19:45 - 22:00 03/08/2016 \newline
\textbf{Location}: Aarhus, Denmark

\section{Hvordan vil et typisk salgsforløb til potentielle kunder forløbe?}
\label{ch:consultantsInterview:flow}
Konsulenten indsamler lidt metadata om en bruger, som navn, alder, dato, erfaring, ketchertype, mv., inden der spilles i en kort periode med systemet. 
Derefter præsenteres den optagede data på en PC. 
Der bør være en umiddelbar spiselig grafisk præsentation med nogle pop-termer, som hvor mange slag, typer og grafer.
Hvis det er interessant for brugeren, så kan der skiftes til nogle mere detaljerede oplysninger.

Efterfølgende vil det være interessant at kunne oplyse brugeren om hvilke muligheder og perspektiver der er for at forbedre sin træning med træningsprogrammer og lignende.

Opfølgning på forløbet kan ske i form af nyhedsbreve og direkte og indirekte henvendelser til klubber og spillere.
Evt. ved fremsendelse af målte data på e-mail og papir.

\section{Hvilke værktøjer er nødvendige i dette forløb?}
\label{ch:consultantsInterview:tools}
En PC, tablet eller lignende hvor der er muligt hurtigt og nemt at præsentere data samt indtaste brugeroplysninger.
Det bør virke uden internetadgang, men skal senere kunne sende data ud til videre behandling.

\section{Hvad skal programmet kunne udføre?}
\label{ch:consultantsInterview:features}

\subsection*{Need to have}
Præsentation af seneste optagede data, sammenholdt mod evt. historisk data.
Helst en grafisk præsentation af historikken.

Fremsendelse af målt data (spillerprofil) på e-mail.

Top-lister og rangeringer mod venner og professionelle.

Skal kunne dele data på sociale medier. 
Gerne direkte fra resultat-e-mailen der er fremsendt.

Bruger oprettelse

\subsection*{Nice to have}
Nyhedsbrevsfremsendelse ud fra en e-mailliste.

Brugere skal kunne indhente achievements når nogle fastsatte måltal er overgåede, f.eks. over 2000 slag.

Sammenligning med andre spillere, hvis de følger dem.

3D illustration af de enkelte slag med kvantitative detaljer undervejs i svingsløjfen.

Træningsplaner til træner for specielle scenarier.

Push-beskeder der kan notificere brugeren om seneste hændelser.

Trænerprofil med overblik over sine spillere og deres fremgang. 

Klubsystem med mange administrative funktioner.
Detaljer kan høres i interviewet fra omkring 48:00. \fxfatal{Add directions to get interview here eventually}
