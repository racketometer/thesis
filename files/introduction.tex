\chapter{Introduction}

\section*{Relevance}
Sports wearables are on the rise with new products launching every month. 
Most large technology companies like \textit{Apple, Inc.} and \textit{Samsung} are investing heavily into the market and \textit{Gartner, Inc.} predicts that ''\textit{From 2015 through 2017, smartwatch adoption will have 48 percent growth(...)}'' \citep{introduction:relevance:gartner}.

New products that enables users to understand their performances and activities are in high demand, but most of the wearables are centered around wristbands and smart phone applications. 
These have certain limitations and the precision of the data gathered can be of poor quality.
The product of this project if different. 
It is in the center of the action and gives precise measurements of what forces act on the racket.
With this in mind the product is different in the technological sense but fits the same user demand of existing wearables.

\section*{Work process}
The project process is managed with the agile SCRUM development framework. 
Tasks are managed in a backlog and pulled into sprints of two weeks. 
The process of adding tasks to the backlog are done in gromming sessions where the next parts of the project is discussed and reflection, i.e., retrospective, of tasks not finished or newly arisen issues are weighted and prioritized.

Visualization of tasks and their state is done with an online scrum board. 
The tool to use is decided in a pre-study, see section \ref{preStudy:scrum}.

There is no delegation of specific tasks, e.g., front-end and back-end. 
Instead tasks are solved as time and prioritization sees fit.
