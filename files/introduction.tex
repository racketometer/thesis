\chapter{Introduction}
This chapter gives a brief introduction to the project and how it is expected to be solved.
The main problem to solve is:

\userstory{Enable athletes to monitor, visualize and track their performances, in an accessible way, when exercising}

The overall platform aims to enable collaboration between athletes, coaches, administration and other stakeholders in the badminton clubs to gain better achievements and progress for the athletes.

The main part is a ''smart'' racket that can provide the club, coach and individual player with a bunch of information about their badminton performances.
This information can help the coaches find weaknesses and show them where to focus their training on a specific player.
The second part is a management system where management staff can handle business such as economics, practice attendance and events such as parties and tournaments.
The system is illustrated in figure \ref{fig:introduction:vision}.

\graphic{1}{rich_vision.png}{The different parts of the project}{fig:introduction:vision}

The vision for this project is to revolutionize club management and progress monitoring.
The progress monitoring will be made possible with a mobile application that can gather data over \textit{Bluetooth} from sensors in the smart racket and transfer it to the management system.
The data will be available on both the mobile device and in the web application for the player, coach and club to analyse.

The club can own smart rackets for their players to use on occasions where the coach gathers the data for both the player and club to see.
The player should also have the option to buy its own smart racket to monitor its progress at every single practice or match.
Data will bind to the player's individual user and will be shareable with the coaches and club.
The players can further more compare their progress and stats with their friends and other players from the club and share training sessions and progress on social media.

\section*{Relevance}
As sports wearables are on the rise with new products launching every month, most large technology companies like \textit{Apple, Inc.} and \textit{Samsung} are investing heavily into the market and \textit{Gartner, Inc.} predicts that ''\textit{From 2015 through 2017, smartwatch adoption will have 48 percent growth(...)}'' \citep{introduction:relevance:gartner}.

New products that enables users to understand their performances and activities are in high demand, but most of the wearables are centered around wristbands and smartphone applications.
These have certain limitations and the precision of the data gathered can be of poor quality.
The product of this project is different.
It is in the center of the action and gives precise measurements of what forces act on the racket.
With this in mind the product is different in the technological sense but fits the same user demands of existing wearables.

\section*{Boundary}
Since the vision is more than two man can overcome during this project's duration, boundaries are set for the specific parts that we want to achieve, figure \ref{fig:introduction:visionBoundaries}.

As mentioned in the relevance section, the part that makes this platform different from a normal club management system, is the progress monitoring through the mobile application and smart racket.
Since this is the unique feature, it is also going to be the focus point for the project.

\graphic{0.8}{rich_vision_with_focus.png}{The parts this project will focus on}{fig:introduction:visionBoundaries}

\section*{External collaboration}
As this is a software project we have allied with two external consultants.
They are responsible for the development of the physical smart racket and the electronics inside.
Further more they have great knowledge of the badminton community and a great personal network with club managers, coaches and players that can be used for testing, research of user demands and alike.
They will be referred to as the ''consultants'' throughout this report.

\section*{Work process}
The project process is managed with the agile Scrum development framework as this is the most popular methodology in software development \citep{introduction:work:scrum}. 
The flow of events utilized in the process are as follows.
A grooming session is held to discuss and define new tasks for the project.
This is a returning event in every sprint and makes sure the \textit{Backlog} is prioritized and up to date.

The \textit{Backlog} is holding all tasks to be carried out.
They are all estimated with Scrum points, an arbitrary value of how large a given task is.
This is an estimate and the team is ''learning by doing'' what a single Scrum point means.
The number of points to burn down each sprint is set by comparing the former sprints total burndown and the expected available resources in the coming sprint.
Each sprint is set for two weeks. 
This was evaluated in the initial weeks of the project where sprints with one and two weeks duration was tested out.
The latter one was what fit us the most as this covered enough work to make progress and little enough to assess.

In the grooming sessions a non-formal \textit{retrospective}, i.e. reflection of the sprint, is held. 
If any tasks were not finished in the current sprint, it is decided if they should be moved into the new sprint or the \textit{Backlog} for latter completion.

Visualization of tasks and their states is done with an online Scrum board.
The tool used was decided in a pre-study, see section \ref{preStudy:scrum}.
On figure \ref{fig:introduction:scrumBoard} is a snapshot of the current sprint.
On the left, column \textit{Ready}, is the tasks ready to be completed in the sprint.
Just left of this column is a collapsed column with the \textit{Backlog}.
Next to this is the \textit{In Progress} column holding all tasks currently being worked upon.
After a task is done it moves to the \textit{Review} column where it is reviewed by a team member.
When the review is done and there are no comments or they are resolved, the tasks is placed in the \textit{Done} column.

To make the management of tasks as easy as possible, the used tool synchronizes with \textit{Git} branches, i.e. in progress work in version control. 
When a new branch is created in \textit{Git}, referencing a task number, it is moved to \textit{In Progress}. 
When a \textit{Pull Request}, i.e. the changes are ready to be merged into the \textit{master} branch, the task is moved to \textit{Review} and finally if the \textit{Pull Request} is closed and merged, the task is put in \textit{Done}.

\graphic{1}{scrumBoard}{Snapshot of the Scrum board}{fig:introduction:scrumBoard}
