\chapter{User survey}
In the initial phase of the project a user survey was distributed over social media to get an understanding of the product user group.
In the period August 29th 2016 to September 20th 2016, the survey received 529 responses of which 425 was users playing badminton.

\subsection*{Racket price}
When comparing the price ranges to user groups it shows that most users have a racket in the price range kr. 1000 - 1499 as seen on figure \ref{fig:survey:priceVsAge}.
It is interesting that in the age of 19 to 24 years users spend the most money on their rackets.
The price ranges can be used to find a proper cost for the product and/or services offered.

\graphic{1}{userSurvey/priceVsAge}{Price distribution versus age groups}{fig:survey:priceVsAge}

\subsection*{Gadget knowledge}
When asking what sport-gadgets the users know, the results show a clear pattern as seen on figure \ref{fig:survey:gadget}.
\Glsi{endomondo} is the most known gadget closely followed by \glsi{gopro} with 87\% and 73\% respectively.
A key difference between these are clear when looking at how many users owning it.
Only 17\% of the users knowing \gls{gopro} owns one whereas it is 52\% with \gls{endomondo}.
An explanation on this could be that \gls{endomondo} has a free version of the application with limited features.
\Glsi{gopro} offers no such thing as it is a physical product.

\graphic{1}{userSurvey/gadget}{Knowledge and ownership of sport-gadgets}{fig:survey:gadget}

\subsection*{Information sharing}
The survey asked the users if they were willing to share their recorded information to improve the algorithms behind the calculated results. 
81\% approved this, figure \ref{fig:survey:share}.
Further more the users was asked if they would share their recordings to match other players for ranking and alike and only 62\% approved this.
As the difference between these two options are quite subtle, i.e. the anonymous part of it, it is clear that this must be taken into account when creating the product.
 
\graphic{1}{userSurvey/share}{Connection of anonymous data to improve algorithm and user data to allow ranking}{fig:survey:share}

\subsection*{Implementation options}
When asking for possible sensor implementation solutions for mounting the sensors to the racket, the most popular option with 57\% of the answers, was mounting a sensor on an existing racket, as seen on figure \ref{fig:survey:implementation}. 
This is the least intrusive solution as well, why this was expected.
The initial approach from the consultants was to mount the sensor inside a racket.
Of the users in the survey, 36\% were willing to buy a new racket with the sensor built in, why this is a viable solution for the prototype.

\graphic{1}{userSurvey/implementation}{Sensor implementation possibilities on racket}{fig:survey:implementation}
