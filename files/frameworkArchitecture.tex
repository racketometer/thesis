\chapter{Framework architecture}

Multiple frameworks are used in the project. 
This is a short description of the responsibilities each framework is responsible for. 
An overview is shown on figure \ref{frameworkArchitecture}.

At the application layer of the front-end \gls{ng2} is responsible for the view rendering and business logic. 
It communicates with the \gls{apollo:client}, that is responsible for communicating with the back-end, as well as holding application state. 
Between the front-end and back-end the \gls{api} is defined with \gls{graphql}.

The client application layer is run with the \gls{nativescript} run-time. 
At compile time, the \gls{ng2} views and styling is rendered to native UI elements and styling. 
The application logic is held in JavaScript and run on a platform specific JavaScript engine, i.e. \gls{v8} for Android and \gls{javascriptcore} for iOS.

On the back-end the application layer is an \gls{apollo:server} instance. 
This is the counter part to the \gls{apollo:client} and provides the reactive data system. 
It further defines the \gls{api} for collecting data from databases like \gls{mongodb}, as this project utilizes.
Both of these frameworks run on the \gls{node} run-time, which in turn runs on the \gls{v8} JavaScript engine.

\graphic{1}{frameworkArchitecture}{Framework architecture overview}{frameworkArchitecture}
