\section{Backends}

This section will evaluate different options to support the project with data storage and possibly for running heavy jobs that you do not what the clients to run. Some \textit{must have} and \textit{nice to have} capabilities is listed below.
	
\textbf{Must have:}
	\begin{itemize}
	\item Be able to store data.
	\item Support authentication.
	\end{itemize}
	
\textbf{Nice to have:}
	
	\begin{itemize}
	\item Generic API.
	\item Reactive data. Updated data in the database should reflect on the client.
	\end{itemize}
	

\subsubsection{Firebase}

\textbf{Function:} Reactive Data storage
\\
\textbf{Database type:} NoSQL
\\
\textbf{Description:}
\\
\textit{Firebase} is a database platform from \textit{Google}. It has a lot of features you do not normally see on a database. One of the important things is authorization. Usually stuff like authorization goes through some sort of Web API before access is allowed to the database. \textit{Firebase} can handle this aspect without a Web API to parent it. It is also a real-time database, which means that the data is reactive. So whenever there is a change in the database, that change is pushed to the clients to keep them up to date at all times. Pushing data into the database is done with JSON. When using \textit{Firebase} you can not run functions or jobs like in other systems, but a company named \textit{Zapier} have made this available through a product of theirs. This is an extra cost and might not be needed.


\subsubsection{Apollo}

\textbf{Function:} Reactive Data storage
\\
\textbf{Database type:} None specific, free to chooose. Client application interface is \textit{GraphQL}.
\\
\textbf{Description:}
\\
\textit{Apollo} or the Apollostack is a modern data stack that offers reactive data. This means that your data will always be up-to-date, if you want it to be, that is. You can choose your poise. If you do not need reactive data you can choose not to. \textit{Apollo} is a data stack provided by the \textit{Meteor} team. As opposed to \textit{Meteor}, \textit{Apollo} is ment to be useable in any application. There are integration guides for \textit{Angular 2}, \textit{Redux}, \textit{Meteor}, \textit{React} and \textit{React native}, but you should be able to integrate it into any JavaScript frontend. \textit{Apollo} uses \textit{GraphQL}, which consists of 2 parts. A Schema part that tells what type of data is available on the server and a query language that enables the client to describe the data it needs. This means that the client have no idea what type of database is behind \textit{GraphQL}. It is comparable to REST, but with \textit{GraphQL} you define what properties you need. This improves data usage cause you do not send unneeded information over the wire.


\subsubsection{Requirements summary}
	
	\begin{tabularx}{\textwidth}{|l|C C|}
	\hline 
	 & Firebase & Apollo \\ 
	\hline 
	\textbf{Be able to store data} & \cmark & \cmark \\ 
	\hline
	\textbf{Support authentication} & \cmark & \cmark \\ 
	\hline 
	Generic API & \xmark & \cmark \\ 
	\hline 
	Reactive data & \cmark & \cmark \\ 
	\hline 	
	\end{tabularx} 
	
\subsubsection{Conclusion}
\textit{Apollo} fulfills all of our requirements and appeals to us for the fact that the client and the server gets seperated and are independant of each other. With \textit{Firebase} the client application would be bound to it. \textit{Apollo} lets us pick what database we want and we can even have several and different databases without the client knowing about it. It provides us with flexibility and seperates the front and the backend, at the cost of more work. The selection on what backend to use therefore falls upon \textit{Apollo}.