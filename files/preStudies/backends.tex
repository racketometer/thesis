\section{Back-end frameworks}
This section will evaluate different frameworks and services to support the project with data storage and possibly for running heavy jobs that are not preferred to run on the client applications.
Some \textit{must have} and \textit{nice to have} capabilities are listed below.
	
\subsection*{Must have}
	\begin{itemize}
	\item Be able to store and retrieve data.
	\item Support authentication.
	\end{itemize}
	
\subsection*{Nice to have}	
	\begin{itemize}
	\item Generic \gls{api}.
	\item Reactive data. Updated data in the database should reflect on the client.
	\end{itemize}
	

\subsection*{Firebase}
\textbf{Function:} Reactive Data storage
\\
\textbf{Database type:} \gls{nosql}
\\
\textbf{Description:}
\\
\textit{Firebase} is a database platform from Google, Inc. 
It has lots of features one do not normally see on a database, e.g. user authentication.
It is a real-time database, i.e. the data is reactive, so whenever there is a change in the database, this is pushed to the clients. 
Pushing data into the database is done with \gls{json} and running specific functionality is not supported.
It is however possible by use of a commercial product from  \textit{Zapier}\fxfatal{Reference to this?}.

\subsection*{Apollo}
\textbf{Function:} Reactive Data storage
\\
\textbf{Database type:} None specific, free to choose.
Client application interface is \textit{\gls{graphql}}.
\\
\textbf{Description:}
\\
\textit{Apollo} is a modern data stack that offers reactive data. 
This means that your data will always be up-to-date if desired.
It is a data stack provided by the \textit{Meteor} team, but as opposed to the \textit{Meteor} framework, \textit{Apollo} is to be used in any application. 

There are integration guides for \textit{\gls{ng2}}, \textit{\gls{redux}}, \textit{Meteor}, \textit{\gls{react}} and \textit{React native}, but one should be able to integrate it into any JavaScript front-end.
\textit{Apollo} consists of two parts. 
A schema that describes the type of data available on the server and a query language, \textit{\gls{graphql}}, that enables the client to describe the data it needs. 
This decouples the client from the database type as \textit{\gls{graphql}} is only a query specification.
\textit{Apollo} can query different types of database, e.g. \textit{\gls{mongodb}} and \textit{\gls{mysql}}, without the client realising it.
As the client specifies what properties are needed, no unused data is send over the wire.

\subsection*{Requirements summary}
This table shows how each framework and service fulfills the specified requirements.
	
	\begin{tabularx}{\textwidth}{|l|C C|}
	\hline 
	 & Firebase & Apollo \\ 
	\hline 
	\textbf{Be able to store and retrieve data} & \cmark & \cmark \\ 
	\hline
	\textbf{Support authentication} & \cmark & \cmark \\ 
	\hline 
	Generic \gls{api} & \xmark & \cmark \\ 
	\hline 
	Reactive data & \cmark & \cmark \\ 
	\hline 	
	\end{tabularx} 
	
\subsection*{Conclusion}
\textit{Apollo} fulfills all of the requirements and provides an excellent separation from the clients and server.
With \textit{Firebase} the client application is bound to it. 
\textit{Apollo} lets the developer pick what database to be used and can support several, and different types, of databases. 
This is at the expense of more work by the developer but a considerable better flexibility why \textit{Apollo} is the selected back-end.