\section{Cross-platform frameworks}

This section will evaluate different frameworks for cross-platform development which could be used in the project. 
Some \textit{must have} and \textit{nice to have} capabilities is listed below.
	
\subsection*{Must have}
	\begin{itemize}
	\item Cross platform capabilities.
	\item Wide community support.
	\item \textit{Bluetooth Low Energy} capabilities.
	\end{itemize}
	
\subsection*{Nice to have}
	
	\begin{itemize}
	\item One platform both for web and mobile applications.
	\item Type safety in the programming language.
	\item Reactive data. Updated data in the database should reflect on the clients.
	\item Compiles to native code for increased performance.
	\end{itemize}
	

\subsection*{Meteor}
\textbf{Language:} JavaScript
\\
\textbf{Description:}
\\
\textit{Meteor} is a platform and framework for developing both web and mobile applications. 
It comes with \textit{Cordova} built-in and is considered super fast for developing applications and prototyping. 
It comes with a lot of built-in tools and a package library for easy setup of things like user  and login integrations of different kinds, e.g. \textit{Google} and \textit{Facebook}. It comes with hot code push which is a tool for uploading new versions of your application to the App-stores.

\textit{Meteor} has its own CLI (Command Line Interface which is very useful when developing and it allows one to write the back-end in plain JavaScript. 
Out of the box it integrates with \textit{Mongo}, a document database for data storage and it uses \textit{DDP} (Distributed Data Protocol) for sending data between the client and the server. 
This data is reactive and will be updated at the clients when it changes.

\textit{Meteor} comes with its own view layer named \textit{Blaze} but this can be substituted with \textit{React} or \textit{AngularJS} if any of them is preferred.

\subsection*{NativeScript}
\textbf{Language:} JavaScript
\\
\textbf{Description:}
\\
\textit{NativeScript} uses a JavaScript toolchain to compile the JavaScript code into native APIs. 
This is a huge advantage because it makes the application work exactly like it was written natively without ever writing any native code or knowing about the native APIs. 
The markup is written in an XML (Extensible Markup Language) like language and is styled with CSS (Cascading Style Sheets).
It is developed by Telerik, a big software company where one can opt-in for support through a paid license. 

It is supported by Google and therefore works well with \textit{AngularJS} and TypeScript and with this the \textit{nice to have} feature ''type safety'' is fulfilled. 
\textit{Bluetooth Low Energy} capabilities can be achieved with the \textit{Node.js} package \textit{nativescript-bluetooth}, created by \textit{npm} user eddyverbruggen. 
\textit{NativeScript} supports \textit{iOS}, \textit{Android}, \textit{Universal Windows Platform} and if \textit{AngularJS} is used it also supports building web applications \citep{preStudy:frameworks:nativescript}.


\subsection*{ReactNative}
\textbf{Language:} JavaScript, Objective-C and Java
\\
\textbf{Description:}
\\
\textit{ReactNative} is a framework by Facebook, Inc.
It extends the \textit{React} framework for building native applications. 
\textit{ReactNative} requires the developer to know Objective-C for \textit{iOS} development and Java for \textit{Android} development. 
This does not fulfill the requirement that it has to be cross-platform without the developer needing to know about the distinct platform languages and APIs.


\subsection*{Xamarin}
\textbf{Language:} C\#
\\
\textbf{Description:}
\\ 
\textit{Xamarin} was recently purchased by Microsoft. It is a cross-platform framework for developing applications for \textit{iOS}, \textit{Android}, \textit{Windows Phone} and \textit{Windows Universal Platform}. 
There are two main ways of developing with \textit{Xamarin}. 
\textit{Xamarin forms} which is the method that has the highest amount of code-sharing between platforms. 
It is estimated around 80-95\% code sharing depending on how specific the UI has to be. 
The other is \textit{Xamarin native} which has around 50-60\% code sharing with mainly business login.
With this the UI implementations differ for each platform. 
When using \textit{Xamarin native} it is a huge advantage to know about the native UI APIs.

\subsection*{Requirements summary}
This table shows how each framework fulfills the specified requirements.
	
	\begin{tabularx}{\textwidth}{|l|C C C C|}
	\hline 
	 & Meteor & NativeScript & ReactNative & Xamarin \\ 
	\hline 
	\textbf{Cross platform} & \cmark & \cmark & \cmark & \cmark \\ 
	\hline 
	\textbf{Community} & \cmark & \cmark & \cmark & \cmark \\ 
	\hline 
	\textbf{BLE capabilities} & \cmark & \cmark & \cmark & \cmark \\ 
	\hline 
	1 platform & \cmark & \cmark & \xmark & \xmark \\ 
	\hline 
	Type safety & \xmark & \cmark & \xmark & \cmark \\ 
	\hline 
	Reactive data & \cmark & \xmark & \xmark & \xmark \\ 
	\hline 
	Native performance & \xmark & \cmark & \cmark & \cmark \\ 
	\hline 
	\end{tabularx} 
	
\subsection*{Conclusion}
\textit{NativeScript} fulfills most of our requirements out of the box. 
The only thing that is not support is reactive data, however this can still be achieved with other tools.
Therefore the selection falls upon \textit{NativeScript}.
