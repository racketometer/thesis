\chapter{Back-end Analysis}
This chapter gives insight into the performance of the back-end application.

\section{Optics}
Optics is an analysis tool build by the \gls{meteor} developer team for use on
\gls{graphql} servers. They currently support \gls{express}, \gls{hapi}, \gls{koa} and \gls{ruby} servers. It is a
commercial product but also contains a free plan for developers.

Optics was added to the project to get insights into the performance of the
different queries the user can perform. The data can then be used to pinpoint
performance bottlenecks and where you need to optimize your application.

\subsection{Performance analysis}
Figure \ref{fig:withoutMeasurement} is a screenshot from Optics that shows the analysis for querying after
the \verb+viewer+, including the \verb+user+ but excluding the users \verb+measurements+.

\graphich{0.72}{withoutMeasurement}{Optics analysis for viewer, including the
user but excluding measurements}{fig:withoutMeasurement}

\newpage

Figure \ref{fig:withoutMeasurement} shows that the average duration for a \verb+viewer+ with
a \verb+user+ only takes 8.19 millisecond to execute. That is pretty fast. But when
asking for the \verb+users measurements+ the duration increases tenfolds, as
illustrated on figure \ref{fig:withMeasurement}

\graphich{1}{withMeasurement_short}{Optics analysis for viewer, including the user and measurements}{fig:withMeasurement}

\newpage

With \verb+measurements+, the query takes on average 147 milliseconds to resolve.
Those statistics are on a \verb+user+ with only a few measurements. From this we can easily see that
when \verb+measurements+ grow we need to change how it is handled. Adding "pagination"
could be one way to solve it. When analyzing closer it becomes clear that the
\verb+data+ field on \verb+measurements+ are where it hurts. When we query without it we get
the result on figure \ref{fig:withMeasurementWithoutData}

Without the \verb+data+ field on \verb+measurements+ the query was resolved in half the
time. This shows the power of \gls{graphql} and proves that asking for data you
don't need can be a big performance penalty.

\graphich{0.9}{withMeasurementWithoutData_short}{Optics analysis for viewer,
including the user and measurements but excluding data}{fig:withMeasurementWithoutData}

\newpage
\subsection{Comparison}
When comparing a few queries side by side it becomes very clear that the amount of data that
is queried have a big impact on performance. Why login is slower than querying after a user
can be tracked down to how the database is setup. Users are only indexed by id. So when login
tries to find a user matching a specific email and password, it takes a lot longer than when
looking for a user with a specific id. By adding indexes to the email, we could see a performance increase. 
\graphich{0.6}{overview_optics}{Comparison of 4 different queries}{fig:overview_optics}
