\chapter{Architecture}
This chapter describes the architecture used in the project.

\section{Domain model}
Given the above introduction a domain model was derived, figure \ref{fig:domainModel}, to make the domain entity relations visible.
This is done to make sure the understanding of the relations are clear before doing any further development.
It is important to note that this is a tentative model and as long as the project is under development, it is subject to change.

\graphic{1}{domain_model}{Domain model}{fig:domainModel}

\section{Framework architecture}
Multiple frameworks are used in the project. 
This is a short description of the tasks each framework is responsible for. 
An overview is shown on figure \ref{fig:frameworkArchitecture}.

At the application layer of the front-end \gls{ng2} is responsible for the view rendering and business logic. 
It communicates with the \gls{apollo:client}, that is responsible for communicating with the back-end, as well as holding application state. 
Between the front-end and back-end the \gls{api} is defined with \gls{graphql}.

The client application layer is run with the \gls{nativescript} run-time. 
At compile time, the \gls{ng2} views and styling is rendered to native UI elements and styling. 
The application logic is held in JavaScript and run on a platform specific JavaScript engine, i.e. \gls{v8} for Android and \gls{javascriptcore} for iOS.

On the back-end the application layer is an \gls{apollo:server} instance. 
This is the counter part to the \gls{apollo:client} and provides the reactive data system. 
It further defines the \gls{api} for collecting data from databases like \gls{mongodb}, as this project utilizes.
Both of these frameworks run on the \gls{node} run-time, which in turn runs on the \gls{v8} JavaScript engine.

\graphic{0.9}{frameworkArchitecture}{Framework architecture overview}{fig:frameworkArchitecture}

\section{Front-end architecture}
The front-end application is written in \gls{ng2}. 
The framework utilizes a component structured \gls{mvc} architecture where each component represents the VC part and services cover M \citep{architecture:ng}, as shown on figure \ref{fig:mvcComponent}.

A component is split into two parts. 
The \verb+template+, specifically in \gls{nativescript}, is written in a language similar to \gls{xml} and represents how a component should be rendered in the \gls{ui}.
It communicates with a \verb+controller+, the component class, with events and get information with property bindings. 
It can communicate with services that are constructor injected by the Angular Dependency Injection system.

A \verb+service+ is responsible for the business logic and can communicate with a back-end or perform calculations etc.
It is possible to do this in the controllers as well, but it makes them harder to unit test and further more ads logic that is not view specific.

\graphic{1}{mvcComponent}{\Gls{mvc} component with \gls{ng2} terminology}{fig:mvcComponent}

The view is build from individual components, each responsible of a clear part of the whole \gls{ui}.
As of this, the complete view can be illustrated as a component tree where each component can propagate events up and down the tree and communicate with services.

On figure \ref{fig:componentTree} the component tree of a simple login view is illustrated.
It contains an input component that uses some validation component and a button.
The root component \verb+Login+ subscribes to click events on the button and change events on the input field.
When the button is clicked, it initiates the authentication routine in the \verb+Authentication+ service.

As the application grows more complex branches of the tree can be modularized.
This makes the structure easy to mangle with and small parts, i.e. components, of the view can easily be reused.

\graphic{0.6}{componentTree}{Component tree of a simple login view}{fig:componentTree}

\section{N+1}

\chapter{Deployment}
This chapter describes how the applications are distributed and deployed and how the production systems are setup.

\section{Deployment model}
The software packages are deployed as a distributed system as shown on figure \ref{fig:deploymentModel}.
The front-end codebase is compiled to native applications on the targeted platforms and deployed to the devices.
The back-end codebase is deployed to a \glsi{heroku} infrastructure running \glsi{node} and is setup to communicate with a \glsi{mongodb} instance hosted on \glsi{mlab}, see section \ref{sec:staging}.

\graphic{1}{deployment}{Software deployment model}{fig:deploymentModel}

\section{Back-end}
The back-end application is deployed on \glsi{heroku} with free hosting. 
This is because the product is still in a beta state and the cheapest solution, i.e free, is chosen.

When deploying a \glsi{node} application on \glsi{heroku}, you deploy your source code from a local git repository.
\Glsi{heroku} then builds the application with \glsi{npm} scripts and starts the application on the running web instance as illustrated on \ref{fig:deploymentFlow}.

\graphic{0.8}{deploymentFlow}{\Glsi{heroku} deployment flow}{fig:deploymentFlow}

\subsection{Existing server}
To deploy to the existing server push git changes to the \glsi{heroku} remote with the following command.

\verb+git push heroku master+

\subsection{New server}
\label{sec:deployment:new}
To deploy the application to a new server follow the \glsi{heroku} guide for deploying a \glsi{node} application on their servers which breaks down the different commands \citep{documentation:deployment:heroku}.

When the application is deployed, the \verb+npm install+ command is automatically run to install the dependencies.
To make sure that \glsi{npm} installs the developer dependencies, the \verb+NPM_CONFIG_PRODUCTION+ environment variable is set to \verb+false+ with the following command in the \glsi{heroku} \gls{cli}.

\verb+heroku config:set NPM_CONFIG_PRODUCITON=false+

\glsi{heroku} is configured with a \verb+Procfile+ \citep{documentation:deployment:heroku:procfile}, with the single command below. This defines how the web application is started.

\verb+web: node dist/server.js+

If no \verb+Procfile+ is provided, \glsi{heroku} will run the \verb+npm start+ command.
That command is used under development and runs a development server.
Therefore a \verb+Procfile+ is provided to run a different command.

To make sure that the source code gets compiled before the server starts, a \verb+postinstall+ script is added to the \verb+package.json+ file.
This script will run the \glsi{webpack} build chain that compiles the \gls{typescript} code and outputs it to the \verb+dist+ folder.

\subsection{Staging}
\label{sec:staging}
As the application uses staging for different environments, several environment variables need to be set.
Specifically the \verb+NODE_ENV+ and the \verb+ROM_DB_PROD+ variable. 
The \verb+NODE_ENV+ variable defines if a database seed should be applied on startup or not. 
This is used on the testing stage so that integration tests are always run against the same data.

In production we do not ever want to reseed our database and therefore the \verb+NODE_ENV+ variable is set to \verb+prod+.

Lastly \verb+ROM_DB_PROD+ needs to be set to the \gls{url} of the production \glsi{mongodb} instance.
For a complete reference on environment variables see chapter \ref{ch:environmentVariables}.

The \glsi{mongodb} for production is hosted on \glsi{mlab} as Database-as-a-Service web service.


\section{Data persistence}
This section describes how data persistence is obtained and also give insight into the database schema shown on figure \ref{fig:dbSchema}.

To persist data this project uses \glsi{mongodb} which is a \gls{documentdb}.
When working with \glspl{documentdb} you often get a more flat and embedded data model than when working with relational databases. 
This is also reflected in our database schema on figure \ref{fig:dbSchema} which only has 2 entities which will translate into 2 collections in the \glsi{mongodb}.

A \gls{node} environment is used on the server to communicate with a standalone \gls{mongodb}.

\subsection{Database schema}
After analyzing the domain and the requirements the following database schema was created to fulfill our needs.

It contains 2 collections i.e. \textit{Measurement} and \textit{User}. \textit{Users} represent all possible users of the system, both players, consultants and coaches. 
They are then distinguish between then from flags on the document (\verb+isConsultant+ and \verb+isCoach+).

\textit{Users} have a zero-to-many relation to itself because users represent every possible users i.e. friends, consultants and coaches.

\textit{Measurement} represents both raw data as well as calculated data.

Measurements have a zero-to-many relationship to users because a user can have 0 measurements, but a measurement will always have at least 1 user and possible more if the measurement was uploaded by a coach or consultant.

\graphic{1}{dbSchema}{MongoDB Schema}{fig:dbSchema}


