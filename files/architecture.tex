\chapter{Architecture}
This chapter describes the architecture used in the project.

\section{Domain model}
Given the above introduction a domain model was derived, figure \ref{fig:domainModel}, to make the domain entity relations visible.
This is done to make sure the understanding of the relations are clear before doing any further development.
It is important to note that this is a tentative model and as long as the project is under development, it is subject to change.

\graphic{1}{domain_model}{Domain model}{fig:domainModel}

\section{Framework architecture}
Multiple frameworks are used in the project. 
This is a short description of the tasks each framework is responsible for. 
An overview is shown on figure \ref{fig:frameworkArchitecture}.

At the application layer of the front-end \gls{ng2} is responsible for the view rendering and business logic. 
It communicates with the \gls{apollo:client}, that is responsible for communicating with the back-end, as well as holding application state. 
Between the front-end and back-end the \gls{api} is defined with \gls{graphql}.

The client application layer is run with the \gls{nativescript} run-time. 
At compile time, the \gls{ng2} views and styling is rendered to native UI elements and styling. 
The application logic is held in JavaScript and run on a platform specific JavaScript engine, i.e. \gls{v8} for Android and \gls{javascriptcore} for iOS.

On the back-end the application layer is an \gls{apollo:server} instance. 
This is the counter part to the \gls{apollo:client} and provides the reactive data system. 
It further defines the \gls{api} for collecting data from databases like \gls{mongodb}, as this project utilizes.
Both of these frameworks run on the \gls{node} run-time, which in turn runs on the \gls{v8} JavaScript engine.

\graphic{0.9}{frameworkArchitecture}{Framework architecture overview}{fig:frameworkArchitecture}

\section{Front-end architecture}
The front-end application is written in \gls{ng2}. 
The framework utilizes a component structured \gls{mvc} architecture where each component represents the VC part and services cover M \citep{architecture:ng}, as shown on figure \ref{fig:mvcComponent}.

A component is split into two parts. 
The \verb+template+, specifically in \gls{nativescript}, is written in a language similar to \gls{xml} and represents how a component should be rendered in the \gls{ui}.
It communicates with a \verb+controller+, the component class, with events and get information with property bindings. 
It can communicate with services that are constructor injected by the Angular Dependency Injection system.

A \verb+service+ is responsible for the business logic and can communicate with a back-end or perform calculations etc.
It is possible to do this in the controllers as well, but it makes them harder to unit test and further more ads logic that is not view specific.

\graphic{1}{mvcComponent}{\Gls{mvc} component with \gls{ng2} terminology}{fig:mvcComponent}

The view is build from individual components, each responsible of a clear part of the whole \gls{ui}.
As of this, the complete view can be illustrated as a component tree where each component can propagate events up and down the tree and communicate with services.

On figure \ref{fig:componentTree} the component tree of a simple login view is illustrated.
It contains an input component that uses some validation component and a button.
The root component \verb+Login+ subscribes to click events on the button and change events on the input field.
When the button is clicked, it initiates the authentication routine in the \verb+Authentication+ service.

As the application grows more complex branches of the tree can be modularized.
This makes the structure easy to mangle with and small parts, i.e. components, of the view can easily be reused.

\graphic{0.6}{componentTree}{Component tree of a simple login view}{fig:componentTree}

\section{Back-end architecture}
The back-end application is written in \gls{typescript} and build with a framework called \gls{apollo}.
The back-end is spun up using \gls{node} and Express which is a framework to spin up a server in a \gls{node} environment.

The back-end contains 3 layers of abstraction which are illustrated on \ref{fig:backendLayer}.

\graphic{1}{backend_architecture}{The back-end architecture layers}{fig:backendLayer}

The top layer is the one that gets exposed to the users of the back-end. The schema consists of models and functions that the user can query for. The Schema is written in the GraphQL Schema Language.

When the user queries, Apollo looks for a resolve function to resolve the query. This layer is called the Resolvers layer. Here, functions are defined to resolve specific queries related to the schema. 

The data access layer contains 2 abstractions. The Connectors are where connections to databases are made and exposed to the above models layer. These databases are used in different ways. In the models layer, these are abstracted away so that the resolvers don't need to change, if the database or the \gls{orm} changes in the connectors layer. The models layer then exposes a generic API for finding, modifying, creating and removing persisted data

Besides the 3 layers that make up the Apollo server, we also have 2 services. One for racket algorithms used to calculate features based on session data and an email service used to send out emails to users when they are signed up or when they change passwords.

\section{N+1}

\section{Deployment model}
The deployment model is meant to illustrate where the different software packages are deployed, as shown on figure \ref{deploymentmodel}.

The \glsi{nativescript} codebase is compiled into 2 different file extensions i.e. \verb+.jar+ for \textit{Android} and \verb+.app+ for \textit{iOS}. The compiled files are then deployed directly to their respective devices or uploaded to each platform's individual \gls{appstore}.

The back-end codebase is deployed on a web server with a \gls{node} environment as well as a \glsi{mongodb}.

\graphic{1}{deployment}{Deployment Model}{deploymentmodel}


\section{Data persistence}
This section describes how data persistence is obtained and also give insight into the database schema shown on figure \ref{fig:dbSchema}.

To persist data this project uses \glsi{mongodb} which is a \gls{documentdb}.
When working with \glspl{documentdb} you often get a more flat and embedded data model than when working with relational databases. 
This is also reflected in our database schema on figure \ref{fig:dbSchema} which only has 2 entities which will translate into 2 collections in the \glsi{mongodb}.

A \gls{node} environment is used on the server to communicate with a standalone \gls{mongodb}.

\subsection{Database schema}
After analyzing the domain and the requirements the following database schema was created to fulfill our needs.

It contains 2 collections, i.e. \textit{Measurement} and \textit{User}. \textit{User}s represent all possible users of the system, both players, consultants and coaches. 
They are then distinguished with flags on the document (\verb+isConsultant+ and \verb+isCoach+).

\textit{Users} have a zero-to-many relation to itself because users represent every possible users i.e. friends, consultants and coaches.

\textit{Measurement} represents both raw data as well as calculated data.

Measurements have a zero-to-many relationship to users because a user can have 0 measurements, but a measurement will always have at least 1 user and possible more if the measurement was uploaded by a coach or consultant.

\graphic{1}{dbSchema}{Database schema with \textit{crow foot} notation}{fig:dbSchema}


