\chapter{Architecture}
This chapter describes the architecture used in the project.

\section{Domain model}
Given the above introduction a domain model was derived, figure \ref{fig:domainModel}, to make the domain entity relations visible.
This is done to make sure the understanding of the relations are clear before doing any further development.
It is important to note that this is a tentative model and as long as the project is under development, it is subject to change.

\graphic{1}{domain_model}{Domain model}{fig:domainModel}

\section{Framework architecture}
Multiple frameworks are used in the project. 
This is a short description of the responsibilities each framework is responsible for. 
An overview is shown on figure \ref{frameworkArchitecture}.

At the application layer of the front-end \gls{ng2} is responsible for the view rendering and business logic. 
It communicates with the \gls{apollo:client}, that is responsible for communicating with the back-end, as well as holding application state. 
Between the front-end and back-end the \gls{api} is defined with \gls{graphql}.

The client application layer is run with the \gls{nativescript} run-time. 
At compile time, the \gls{ng2} views and styling is rendered to native UI elements and styling. 
The application logic is held in JavaScript and run on a platform specific JavaScript engine, i.e. \gls{v8} for Android and \gls{javascriptcore} for iOS.

On the back-end the application layer is an \gls{apollo:server} instance. 
This is the counter part to the \gls{apollo:client} and provides the reactive data system. 
It further defines the \gls{api} for collecting data from databases like \gls{mongodb}, as this project utilizes.
Both of these frameworks run on the \gls{node} run-time, which in turn runs on the \gls{v8} JavaScript engine.

\graphic{0.9}{frameworkArchitecture}{Framework architecture overview}{frameworkArchitecture}

\section{N+1}

\section{Deployment model}
The deployment model is meant to illustrate where the different software packages are deployed, as shown on figure \ref{deploymentmodel}.

The \glsi{nativescript} codebase is compiled into 2 different file extensions i.e. \verb+.jar+ for \textit{Android} and \verb+.app+ for \textit{iOS}. The compiled files are then deployed directly to their respective devices or uploaded to each platform's individual \gls{appstore}.

The back-end codebase is deployed on a web server with a \gls{node} environment as well as a \glsi{mongodb}.

\graphic{1}{deployment}{Deployment Model}{deploymentmodel}


\section{Data persistence}
This section describes how data persistence is obtained and also give insight into the database schema shown on figure \ref{fig:dbSchema}.

To persist data this project uses \glsi{mongodb} which is a \gls{documentdb}.
When working with \glspl{documentdb} you often get a more flat and embedded data model than when working with relational databases. 
This is also reflected in our database schema on figure \ref{fig:dbSchema} which only has 2 entities which will translate into 2 collections in the \glsi{mongodb}.

A \gls{node} environment is used on the server to communicate with a standalone \gls{mongodb}.

\subsection{Database schema}
After analyzing the domain and the requirements the following database schema was created to fulfill our needs.

It contains 2 collections, i.e. \textit{Measurement} and \textit{User}. \textit{User}s represent all possible users of the system, both players, consultants and coaches. 
They are then distinguished with flags on the document (\verb+isConsultant+ and \verb+isCoach+).

\textit{Users} have a zero-to-many relation to itself because users represent every possible users i.e. friends, consultants and coaches.

\textit{Measurement} represents both raw data as well as calculated data.

Measurements have a zero-to-many relationship to users because a user can have 0 measurements, but a measurement will always have at least 1 user and possible more if the measurement was uploaded by a coach or consultant.

\graphic{1}{dbSchema}{Database schema with \textit{crow foot} notation}{fig:dbSchema}


