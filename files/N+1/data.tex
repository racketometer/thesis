\section{Data persistence}
This section describes how data persistence is obtained and also give insight into the database schema shown on figure \ref{fig:dbSchema}.

To persist data this project uses \glsi{mongodb} which is a \gls{documentdb}.
When working with \glspl{documentdb} you often get a more flat and embedded data model than when working with relational databases. 
This is also reflected in our database schema on figure \ref{fig:dbSchema} which only has 2 entities which will translate into 2 collections in the \glsi{mongodb}.

A \gls{node} environment is used on the server to communicate with a standalone \gls{mongodb}.

\subsection{Database schema}
After analyzing the domain and the requirements the following database schema was created to fulfill our needs.

It contains 2 collections i.e. \textit{Measurement} and \textit{User}. \textit{Users} represent all possible users of the system, both players, consultants and coaches. 
They are then distinguish between then from flags on the document (\verb+isConsultant+ and \verb+isCoach+).

\textit{Users} have a zero-to-many relation to itself because users represent every possible users i.e. friends, consultants and coaches.

\textit{Measurement} represents both raw data as well as calculated data.

Measurements have a zero-to-many relationship to users because a user can have 0 measurements, but a measurement will always have at least 1 user and possible more if the measurement was uploaded by a coach or consultant.

\graphic{1}{dbSchema}{MongoDB Schema}{fig:dbSchema}
