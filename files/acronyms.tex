% Define acronyms and glossaries here.

%
% Glossaries
%
% \newglossaryentry{<label>}{
%     name = {<name>},
%     description = {<description>}
% }

\newglossaryentry{ng1}{
    name = {AngularJS},
    description = {A JavaScript Single-Page application framework from Google}
}

\newglossaryentry{ng2}{
    name = {Angular 2},
    description = {A JavaScript Single-Page application framework from Google}
}

\newglossaryentry{react}{
    name = {React},
    description = {A JavaScript Single-Page application framework from Facebook}
}

\newglossaryentry{redux}{
    name = {Redux},
    description = {A JavaScript framework to enable a state container in an application}
}
    
\newglossaryentry{bluetooth}{
    name = {Bluetooth},
    description = {A wireless technology standard to transfer data over short ranges between devices}
}

\newglossaryentry{bluetooth:ble}{
    name = {Bluetooth Low Energy},
    description = {A wireless technology standard based on Bluetooth but with lower power consumption}
}

\newglossaryentry{scrum}{
    name = {Scrum},
    description = {Agile software development framework}
}

\newglossaryentry{scrum:board}{
    name = {Scrum board},
    description = {A board consisting of columns named after task states. Tasks move from left to right over the board an visualize the progress of the current sprint}
}

\newglossaryentry{scrum:point}{
    name = {Scrum point},
    description = {An arbitrary measure of the effort to complete a Scrum task}
}

\newglossaryentry{cpd}{
    name = {Cross platform development},
    description = {Software development that targets multiple operating systems and hardware platforms, e.g. \textit{Android} and \textit{iOS} operating systems}
}

\newglossaryentry{userstory}{
    name = {user story},
    plural = {user stories},
    description = {Sentence describing a given user demand in the form: As a <user>, I want to <action>}
}

\newglossaryentry{typedefinition}{
    name = {type definition},
    description = {TypeScript files translating JavaScript sources to TypeScript sources}
}

\newglossaryentry{cordova}{
    name = {Cordova},
    description = {An open source framework for developing cross-platform mobile applications with \glstext{html}, \glstext{css} and JavaScript}
}

\newglossaryentry{ddpg}{
    name = {Distributed Data Protocol},
    description = {A client-server protocol based on the publish-subscribe messaging pattern to synchronize changes to the backend database with clients}
}

\newglossaryentry{apig}{
    name = {API},
    description = {Interface between applications to enable well defined communication methods}
}

\newglossaryentry{node}{
    name = {Node.js},
    description = {A JavaScript runtime running on multiple platforms that enable developers to write applications in JavaScript without the need of a browser}
}

\newglossaryentry{nosql}{
    name = {NoSQL},
    description = {Database based on a non-relational data schema}
}

\newglossaryentry{mysql}{
    name = {MySQL},
    description = {Open Source database based on a relational data schema}
}

\newglossaryentry{mongodb}{
    name = {MongoDB},
    description = {Open Source document database based on a non-relational data schema}
}

\newglossaryentry{graphql}{
    name = {GraphQL},
    description = {A query language specification defineing the \gls{api} between destributed software systems}
}

%
% Acronyms
%
% \newacronym{<label>}{<name>}{<description>}
%
% or if reference to descibtion is nessesary:
%
% \newglossaryentry{ble}{
%     type = \acronymtype,
%     name = {<name>},
%     description = {<description>},
%     first = {<description> (<name>))\glsadd{<labelToGlossary>}},
%     see = [Glossary:]{<labelToGlossary>}
% }

\newacronym{oss}{OSS}{Open Source Software}
\newacronym{html}{HTML}{Hyper Text Markup Language}
\newacronym{css}{CSS}{Cascading Style Sheets}
\newacronym{cli}{CLI}{Command Line Interface}
\newacronym{xml}{XML}{Extensible Markup Language}
\newacronym{ui}{UI}{User Interface}
\newacronym{json}{JSON}{JavaScript Object Notation}

\newglossaryentry{ble}{
    type = \acronymtype,
    name = {BLE},
    description = {Bluetooth Low Energy},
    first = {Bluetooth Low Energy (BLE)\glsadd{bluetooth:ble}},
    see = [Glossary:]{bluetooth:ble}
}

\newglossaryentry{ddp}{
    type = \acronymtype,
    name = {DDP},
    description = {Distributed Data Protocol},
    first = {Distributed Data Protocol (DDP)\glsadd{ddpg}},
    see = [Glossary:]{ddpg}
}

\newglossaryentry{api}{
    type = \acronymtype, 
    name = {API}, 
    description = {Application Programming Interface},
    first = {Application Programming Interface (API)\glsadd{apig}},
    see = [Glossary:]{apig}
}
