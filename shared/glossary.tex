% Define acronyms and glossaries here.

%
% Glossaries
%
% \newglossaryentry{<label>}{
%     name = {<name>},
%     description = {<description>}
% }

\newglossaryentry{ng1}{
    name = {AngularJS},
    description = {A JavaScript Single-Page application framework from Google}
}

\newglossaryentry{ng2}{
    name = {Angular 2},
    description = {A JavaScript Single-Page application framework from Google}
}

\newglossaryentry{apollo}{
    name = {Apollo},
    description = {A JavaScript data transfer framework with reactive data sharing}
}

\newglossaryentry{apollo:client}{
    name = {Apollo Client},
    description = {A part of the \glsi{apollo} framework managing the communication with the back-end application}
}

\newglossaryentry{apollo:server}{
    name = {Apollo Server},
    description = {A part of the \glsi{apollo} framework managing the communication with the front-end application}
}

\newglossaryentry{apollo:optics}{
    name = {Optics},
    description = {Tool for performance analysis on \glsi{graphql} servers}
}

\newglossaryentry{typescript}{
    name = {TypeScript},
    description = {A superset of JavaScript. It is a strongly typed language that compiles to JavaScript}
}

\newglossaryentry{react}{
    name = {React},
    description = {A JavaScript Single-Page application framework from Facebook}
}

\newglossaryentry{redux}{
    name = {Redux},
    description = {A JavaScript framework to enable a state container in an application}
}

\newglossaryentry{bluetooth}{
    name = {Bluetooth},
    description = {A wireless technology standard to transfer data over short ranges between devices}
}

\newglossaryentry{bluetooth:ble}{
    name = {Bluetooth Low Energy},
    description = {A wireless technology standard based on Bluetooth but with lower power consumption},
    symbol = {BLE}
}

\newglossaryentry{scrum}{
    name = {Scrum},
    description = {Agile software development framework}
}

\newglossaryentry{scrum:board}{
    name = {Scrum board},
    description = {A board consisting of columns named after task states. Tasks move from left to right over the board an visualize the progress of the current sprint}
}

\newglossaryentry{scrum:point}{
    name = {Scrum point},
    description = {An arbitrary measure of the effort to complete a Scrum task}
}

\newglossaryentry{cpd}{
    name = {Cross platform development},
    description = {Software development that targets multiple operating systems and hardware platforms, e.g. \textit{Android} and \textit{iOS} operating systems}
}

\newglossaryentry{userstory}{
    name = {user story},
    plural = {user stories},
    description = {Sentence describing a given user demand in the form: As a <user>, I want to <action>}
}

\newglossaryentry{typedefinition}{
    name = {type definition},
    description = {TypeScript files translating JavaScript sources to TypeScript sources}
}

\newglossaryentry{nativescript}{
    name = {NativeScript},
    description = {A TypeScript framework and build-tool set to develop cross-platform mobile applications and compile to native sources}
}

\newglossaryentry{cordova}{
    name = {Cordova},
    description = {An open source framework for developing cross-platform mobile applications with \glstext{html}, \glstext{css} and JavaScript}
}

\newglossaryentry{ddp}{
    name = {Distributed Data Protocol},
    description = {A client-server protocol based on the publish-subscribe messaging pattern to synchronize changes to the backend database with clients}
}

\newglossaryentry{api}{
    name = {API},
    description = {Interface between applications to enable well defined communication methods}
}

\newglossaryentry{mvc}{
    name = {MVC},
    description = {Software architecture where the application responsibilities are separated into Model, responsible for business logic, View defining the UI and Controller facilitating communications between the view and models.}
}

\newglossaryentry{node}{
    name = {Node.js},
    description = {A JavaScript run-time running on multiple platforms that enable developers to write applications in JavaScript without the need of a browser}
}

\newglossaryentry{npm}{
    name = {npm},
    description = {\Glsi{node} package manager for JavaScript dependencies}
}

\newglossaryentry{github}{
    name = {GitHub},
    description = {An online platform for software collaboration through Git version control}
}

\newglossaryentry{javascriptcore}{
    name = {JavaScriptCore},
    description = {The JavaScript run-time on iOS devices}
}

\newglossaryentry{v8}{
    name = {V8},
    description = {A JavaScript run-time for Android devices and browsers}
}

\newglossaryentry{nosql}{
    name = {NoSQL},
    description = {Database based on a non-relational data schema}
}

\newglossaryentry{mysql}{
    name = {MySQL},
    description = {Open Source database based on a relational data schema}
}

\newglossaryentry{mongodb}{
    name = {MongoDB},
    description = {Open Source document database based on a non-relational data schema}
}

\newglossaryentry{travis}{
    name = {Travis CI},
    description = {Continuous integration server as a service for testing and deploying your application}
}

\newglossaryentry{graphql}{
    name = {GraphQL},
    description = {A query language specification defining the \gls{api} between distributed software systems}
}

\newglossaryentry{graphiql}{
    name = {GraphiQL},
    description = {A graphical interface of the \gls{graphql} Schema, usable in a browser both for viewing the schema, but also for querying}
}

\newglossaryentry{documentdb}{
    name = {document database},
    description = {A document database is a non-relational (\gls{nosql}) databases designed to store semi-structured data as documents (rows)}
}

\newglossaryentry{appstore}{
    name = {app store},
    description = {A store for applications on a device i.e. \textit{Google Play} on Android and \textit{App Store} on iOS}
}

\newglossaryentry{endomondo}{
    name = {Endomondo},
    description = {Mobile application acting as a personal trainer. It tracks movement, helps with training schedules and other training related tasks}
}

\newglossaryentry{gopro}{
    name = {GoPro},
    description = {Small HD action camera build to withstand most environments}
}

\newglossaryentry{meteor}{
    name = {Meteor},
    description = {A JavaScript platform for building web, desktop and mobile applications}
}

\newglossaryentry{express}{
    name = {Express},
    description = {A minimalistic web framework for \gls{node}}
}
    
\newglossaryentry{koa}{
    name = {Koa},
    description = {Next generation web framework for \gls{node} made by the Express team}
}

\newglossaryentry{hapi}{
    name = {Hapi},
    description = {Rich web framework for \gls{node}}
}

\newglossaryentry{ruby}{
    name = {Ruby},
    description = {Programming language with focus on simplicity and productivity}
}

\newglossaryentry{webpack}{
    name = {Webpack},
    description = {Module bundler for JavaScript applications}
}

\newglossaryentry{babel}{
    name = {Babel},
    description = {JavaScript compiler for next versions of JavaScript}
}

\newglossaryentry{repl}{
    name = {REPL},
    description = {Execution environment based on "read-eval-print loop" pattern to execute compilable code instantaneous}
}

%
% Acronyms
%
% \newacronym{<label>}{<name>}{<description>}
%
% or if reference to description is nessesary:
%
% \newacronym[see={[Glossary:]{<labelToGlossary}}]{<reference>}{<name>}{<description>}

\newacronym{oss}{OSS}{Open Source Software}
\newacronym{html}{HTML}{Hyper Text Markup Language}
\newacronym{css}{CSS}{Cascading Style Sheets}
\newacronym{cli}{CLI}{Command Line Interface}
\newacronym{xml}{XML}{Extensible Markup Language}
\newacronym{ui}{UI}{User Interface}
\newacronym{json}{JSON}{JavaScript Object Notation}
\newacronym{orm}{ORM}{Object-relational mapping}
\newacronym{http}{HTTP}{Hyper Text Transport Protocol}
\newacronym{smtp}{SMTP}{Simple Mail Transport Protocol}
\newacronym{uuid}{UUID}{Universally unique identifier}
