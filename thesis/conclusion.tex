\chapter{Conclusion}
This bachelor project was commenced four month ago.
The goal was to develop an application platform for badminton athletes, coaches and consultants that enable them to monitor and track their performances.
This was done by developing a cross-platform mobile application for \textit{Android} and \textit{iOS} and a server application to handle data storage.

When defining requirements for the product, we distributed a user survey with social media and collected responses from $529$ individuals.
This gave us an incredible insight into the user segments.
Combined with the consultant interview we have clear understanding of what to do and what \textit{not} to do.
This was a great experience to validate our expected requirements with those of the users.

The development workflow has been a great success.
We have exercised modern development methodologies and had great benefits of working agile and systematically.
From task grooming, sprint planning and retrospection made the work progress clear and understandable.
Task reviews before merging \textit{Git} branches helped keeping the code quality high as well as it gave the individual developers understanding of the overall structure and movements of the code.

In the process of development, it was decided to focus on new technologies, aware of the added risks, for the technology stacks.
This caused us to have several problems with the front-end application framework \glsi{nativescript} with bad documentation, non-compliant build and test configurations and more.
This was a real setback on the technology choice when considering the importance of tests and documentation.

On the back-end, the technologies were more stable and proved to be a great success.
The \gls{api} specification \glsi{graphql} is a powerful concept with fast development and auto-generated interactive documentation.
It enabled us to develop the back-end with a clear structure and let it be ignorant of the client uses.

The final applications did not meet all of our requirements and we have identified several factors why this is the case.
The development of the racket sensor was not completed in time, why all interaction with the racket and views regarding these actions has not been completed.
Further more, the expected amount of views on the front-end application did not prove accurate. 
Several other views and navigation management was needed, which we did not account for when the roadmap was created.
The above, combined with too high ambitions made the product considerably less functional than the requirements, but we are still satisfied with the results.
We have made much of the initial work and it is, compared to the work done, a fairly small task to finish the missing views.

The mobile application was not deployed to the \textit{iOS} \glsi{appstore}, as there are several weeks of delay when getting an app review from Apple.
Further more, a developer license costs a lot of money, which we will not spend just test the application after an \textit{App Store} download.
