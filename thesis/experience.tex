\chapter{Experience}
In the process of developing the applications and working with the domain we have made some valuable experiences.

Overall, the system engineering challenge of grasping the complete domain, breaking it down into consumable parts and understanding the needs of each of these is a challenging task.
We have made use of the methods used by modern software development teams and combined it with the needs of this thesis' educational requirements and this has proven very useful.

The \glsi{scrum} methodology have given us an incredible insight into how we estimate and define tasks, while keeping the overall goal in sight.
Combined with the version control software this has given us the ability to work with the same flows and terms as if it was a professional development team with release notes, pull requests, reviews and more.

We have used user surveys and interviews to understand the requirements of the products and used this to prioritize what tasks to do first.
Further more, we have used the same consultants to look into application flow mock ups and direct us to do make the right choices.

An important lesson that we learned was to be more specific when we set requirements to our collaborators.
The hardware for the rackets was not completed in time and simple misunderstandings of what the minimum requirements were, have delayed the process of developing the sensors.
From this, we now know that we must be as specific with our own requirements to collaborators as we are when setting up requirements for our own software.

All these things combined have learned us how important it is to have a transparent and understandable work flow as well as clear requirements to what needs to be done.

On the more technical parts, we have learned that sales sites tend to show a bright picture of what a framework can accomplish.
We were sure \glsi{nativescript} was the right choice for us when ending the pre-study on cross-platform frameworks, section \ref{sec:crossplatform}, but when getting into the development phase of things, it turned out to be much more cumbersome and tedious to get good results.
We were not able to set up testing on device emulators and had problems with getting the included tooling like \textit{hot reload} of the application to work.
As testing is an important part of improving and maintaining software quality, we set up a stand alone testing environment. that made functional tests of the code.

On the other hand, we have only been positively impressed with how clear and straight to the point the \glsi{graphql} specification allows for development, and how the \glsi{apollo} framework enables high productivity when writing the software.
It was fast and easy and gave us impressive functionality with relatively low amount of code.
