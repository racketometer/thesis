\section{Back-end frameworks}
This section will evaluate different frameworks and services to support the project with data storage and possibly for running heavy jobs that are not preferred to run on the client applications.

As a minimum, the back-end must have capabilities to store and retrieve data from a database and access should be limited with authentication.
In the ''nice to have'' part, an implementation independent interface, i.e. a generic \gls{api}, for the clients is preferred.
This makes sure the clients do not become dependent on the back-end implementation.
As with the cross-platform frameworks, the back-end must support some kind of reactive data structure to follow modern standards.
This means that whenever data changes on the back-end clients are notified.

The \textit{must have} and \textit{nice to have} capabilities are summed up below.

\subsection*{Must have}
\begin{itemize}
	\item Be able to store and retrieve data.
	\item Support authentication.
\end{itemize}

\subsection*{Nice to have}
\begin{itemize}
	\item Generic \gls{api}.
	\item Reactive data. Updated data in the database should reflect on the client.
\end{itemize}

\subsection*{Firebase}
\textbf{Function:} Reactive Data storage
\\
\textbf{Database type:} \gls{nosql}
\\
\textbf{Description:}
\\
\textit{Firebase} is a database service from Google, Inc.
It has lots of features one do not normally see on a database, e.g. user authentication.
It is a real-time database, i.e. data is reactive.
Pushing data into the database is done with \gls{json} and running specific functionality is not supported.

\subsection*{Apollo}
\textbf{Function:} Reactive Data storage
\\
\textbf{Database type:} None specific, free to choose.
Client application interface is \glsi{graphql}.
\\
\textbf{Description:}
\\
\Glsi{apollo} is a modern data stack that offers reactive data.
It is a data stack provided by the \glsi{meteor} team, but as opposed to the \glsi{meteor} framework, \glsi{apollo} is to be used in any application.

There are integration guides for \glsi{ng2}, \glsi{redux}, \glsi{meteor}, \glsi{react} and \textit{React native}, but one should be able to integrate it into any JavaScript front-end.
\glsi{apollo} consists of two parts.
A schema that describes the type of data available on the server and a query language, \glsi{graphql}, that enables the client to describe the data it needs.
This decouples the client from the database type as \glsi{graphql} is only a query specification.
\glsi{apollo} can query different types of databases, e.g. \glsi{mongodb} and \glsi{mysql}, without the client realising it.
As the client specifies what properties are needed, no unused data is send over the wire.

\subsection*{Requirements summary}
This table shows how each framework and service fulfils the specified requirements.

\begin{tabularx}{\textwidth}{|l|C C|}
	\hline
	 & Firebase & \glsi{apollo} \\
	\hline
	\textbf{Be able to store and retrieve data} & \cmark & \cmark \\
	\hline
	\textbf{Support authentication} & \cmark & \cmark \\
	\hline
	Generic \gls{api} & \xmark & \cmark \\
	\hline
	Reactive data & \cmark & \cmark \\
	\hline
\end{tabularx}

\subsection*{Conclusion}
\Glsi{apollo} fulfils all of the requirements and provides an excellent separation from the clients and server.
With \textit{Firebase} the client application is bound to it.
\Glsi{apollo} lets the developer pick what database to be used and can support several, and different types, of databases.
This is at the expense of more work by the developer but a considerable better flexibility why \glsi{apollo} is the selected back-end.
