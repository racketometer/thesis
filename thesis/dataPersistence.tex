\section{Data persistence}
This section describes how data persistence is obtained and also give insight into the database schema shown on figure \ref{fig:dbSchema}.

To persist data this project uses \glsi{mongodb} which is a \gls{documentdb}.
When working with \glspli{documentdb} you often get a more flat and embedded data model than when working with relational databases. 
This is also reflected in our database schema on figure \ref{fig:dbSchema} which only has two entities which will translate into two collections in the \glsi{mongodb}.

The back-end application communicates with a \glsi{mongodb} instance hosted with \glsi{mlab} when running in production.
A local instance can be used when running development by use of environment variables, see section \ref{sec:staging} for details.

\subsection{Database schema}
After analysing the domain and the requirements the following database schema was created to fulfill our needs.

It contains 2 collections, i.e. \textit{Measurement} and \textit{User}. \textit{User}s represent all possible users of the system, both players, consultants and coaches. 
They are then distinguished with flags on the document (\verb+isConsultant+ and \verb+isCoach+).

\textit{Users} have a zero-to-many relation to itself because users represent every possible users i.e. friends, consultants and coaches.

\textit{Measurement} represents both raw data as well as calculated data.

Measurements have a zero-to-many relationship to users because a user can have 0 measurements, but a measurement will always have at least 1 user and possible more if the measurement was uploaded by a coach or consultant.

\graphic{1}{dbSchema}{Database schema with \textit{crow foot} notation}{fig:dbSchema}
