\chapter{Requirements}\label{ch:Requirements}
This chapter defines the requirements for the project.
The functional requirements are described with \glspli{userstory} with some additional conditions of satisfaction to increase the testable details and add non-functional requirements.

\section*{User stories}

\userstory{As a user, I want to use a phone or tablet to access the system}

This \glsi{userstory} is specified in the consultant interview appendix \ref{app:ch:consultantsInterview:tools}.

Device operating systems to support include: \newline
\begin{tabularx}{\textwidth}{X}
    \textit{Apple iOS} 9 and 10 \\
    \textit{Android} 4.3, 4.4, 5.0, 5.1, 6.0 and 7.0
\end{tabularx}

This covers 92.4\% and 92\% of \textit{Android} and \textit{iOS} devices respectively, according to the distribution tables in appendices \ref{app:ch:androidDistribution} and \ref{app:ch:appleDistribution}.

\userstory{As a consultant / coach, I want to input user information when beginning a session and automatically add them to the user database}

This \glsi{userstory} is specified in the consultant interview appendix \ref{app:ch:consultantsInterview:flow}.

If Internet is present, the user information must be submitted immediately to the database.

If Internet is not present, the user information must be stored on the device for later automatic submission.

Information to input must include the following fields:\newline
\field[The full name of the user]{name}{string}
\field{email}{string}
\field{birthday}{Date}
\field[Years of experience]{experience}{integer}
\field[Allow comparison to other users]{isComparisonAllowed}{boolean}
\field[User type]{isCoach}{boolean}

\userstory{As a consultant, I want to start a session with an athlete and record data from the racket}

This \glsi{userstory} is specified in the consultant interview appendix \ref{app:ch:consultantsInterview:flow}.

Information is transmitted from racket to the system with \glsi{bluetooth:ble}.

Information saved for a session must include: \newline
\field{consultant}{Consultant}
\field[Date of the recording session]{date}{Date}
\field{location}{Location}
\field[Recorded racket data]{racketDate}{RacketData}
\field{user}{User}

\userstory{As a consultant, I want started sessions to be persisted between application restarts}

This is an expected behaviour and is therefore added as a requirement.

\userstory{As a consultant / coach, I want to find a previously stored user}

This \glsi{userstory} is specified in the consultant interview appendix \ref{app:ch:consultantsInterview:flow}.

If Internet is present, the list of stored users is updated to reflect the users stored.

If no Internet is present, the cached list of users is listed.

\userstory{As a consultant, I want to show details of recorded data to specific users}

This \glsi{userstory} is specified in the consultant interview appendix \ref{app:ch:consultantsInterview:flow} and \ref{app:ch:consultantsInterview:features}.

Measured data to be presented as numbers or graphs: \newline
\field[Number of strokes]{strokes}{integer}
\field[Most used types of strokes from predefined types]{types}{Array<StrokeType>}
\field[Maximum racket speed in m/s]{maxRacketSpeed}{double}
\field[Maximum shuttlecock speed in m/s]{maxShuttleCockSpeed}{double}

\userstory{As a user, I want to receive an email with an auto-generated password for my account, when a consultant / coach have created an account on my behalf}

This \glsi{userstory} is specified in the consultant interview appendix \ref{app:ch:consultantsInterview:features}.

When a consultant / coach creates an account on behalf of a user, the user should be able to access that account with an auto-generated password send to its account after creation.

If Internet is present, the email must be submitted immediately.

If Internet is not present, the email must be stored in the system for later automatic submission.

\userstory{As a user, I want to be able to change my auto-generated password from within the application}

If Internet is not present, this operation is not possible.

\userstory{As an athlete, I want to compare my performance data to other athletes}

This \glsi{userstory} is specified in the consultant interview appendix \ref{app:ch:userSurvey}.

Athletes that allow comparison must be able to be matched on ranking lists.

Ranking lists must be available on the following metrics: \newline
\begin{tabularx}{\textwidth}{X}
    Number of total strokes \\
    Number of strokes on specific predefined stroke types \\
    Racket speed \\
    Shuttlecock speed \\
\end{tabularx}

\userstory{As a user, I want to share my user profile on social media}

This \glsi{userstory} is specified in the consultant interview appendix \ref{app:ch:userSurvey}.

Social media includes Facebook and Twitter.

The shared information must be a summary of the performance, optionally edited by the user, and a link to the complete user profile on the web page.

\userstory{As an athlete, I want to see my recorded data compared to historical data}

This \glsi{userstory} is specified in the consultant interview appendix \ref{app:ch:userSurvey}.

If an athlete has multiple recorded sessions, metrics must be presented in comparison with the previously recorded metrics.
