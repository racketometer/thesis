\chapter{Results}
There are three major software results of this project, the two mobile applications and the back-end server application.
The mobile applications are written in the cross-platform framework \glsi{nativescript} and are native applications with a shared code base, see figure \ref{fig:results:screenshot}.
It is deployed and available on the \textit{Android} app store \textit{Google Play}.
The applications provide multiple features.

\graphic{1}{appScreenshot}{Screenshot of the login screen on a \textit{Nexus 10} emulator running \textit{Android} $6.0$}{fig:results:screenshot}

\begin{itemize}
\item User login - authentication against back-end application
\item User type specific dashboards - different view if user is consultant / coach or athlete
\item \textit{New user} functionality - for consultants and coaches
\item User list – overview of users created by current user
\item Forced password change - for users with auto-generated passwords
\item Communication with peripheral devices over \glsi{bluetooth:ble}
\item Server communication compliant with the \glsi{graphql} specification over the Internet
\item Cross-platform application with maximum code-share - only minor adjustments of view styles to match the native platforms
\end{itemize}

The back-end application is a \glsi{node} application deployed to \glsi{heroku}.
It is capable of handling several features.

\begin{itemize}
\item \Gls{api} protection - with authentication tokens
\item Web \glspl{api} - for collecting user and measurement data, creating users and changing passwords
\item Web \gls{api} - compliant with the \glsi{graphql} specification
\item Email notifications - with auto-generated passwords for new users, created by consultants and coaches
\item Measurement analysis algorithms - to process racket sensor data
\item Database communication - with the \glsi{documentdb} \glsi{mongodb}
\end{itemize}

In terms of the development setup, it consists of a staging environment for development, testing and production, handling database connections and seeding.
There is a \glsi{ci} server running on all repositories with automated builds, tests and code coverage calculations.
Unit tests are running on both applications and integration tests are testing the web \glspl{api} on the back-end.
