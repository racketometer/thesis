\chapter{Testing}
This chapter describes how testing is set up and managed a cross the applications.

\section{Front-end}
\Glsi{nativescript} have built in unit testing functionality to run tests on hardware or emulated devices \citep{testing:nativescript}.
When setting this up, there were, for now, an unfixable complication caused by the front-end application build configuration.
The project utilizes \glsi{webpack} to bundle and manipulate the source code at compile time and this is not compatible with the  unit testing flow of \glsi{nativescript}.

As of this, we decided to test the business logic of the application with a stand-alone testing framework running on \glsi{node}.
As \glsi{node} runs on the \glsi{v8} run-time, same as \textit{Android}, it still provides valid results though not running the tests on actual devices.

The business logic is only unit tested as integration tests are not possible without the \glsi{nativescript} testing framework.

\section{Back-end}
The back-end application is tested on two levels.
The first level is unit tests.
These are run on isolated functions where dependencies are mocked away.
The second level is integration tests.
These test the application on every level, with nothing mocked away. 

\subsubsection{Staging}
A staging environment is used to make sure that tests are run independently of the production environment. 
Two environment variables are important in this context.
\verb+NODE_ENV+ and \verb+ROM_DB_TEST+ are set on the \glsi{travis} server to make sure that the right things happen.

When \verb+NODE_ENV+ is set to \verb+test+, the database will be seeded upon server start.
That should only ever happen in a testing or developer environment and never in a production environment.

To avoid reseeds of the production database, two different environment variables are used.
When \verb+NODE_ENV+ is set to \verb+test+, the \verb+ROM_DB_TEST+ URL is used. 
When \verb+NODE_ENV+ is set to \verb+prod+, the \verb+ROM_DB_PROD+ URL is used. 
When running tests locally, \verb+NODE_ENV+ is automatically set to \verb+test+.

If needed, the developer can run his code locally against the production database, but will never accidentally reseed the database.

\subsubsection{Unit tests}
Unit tests, test isolated parts of the application. 
They test functions at the different layers, where the below layer will be mocked away to make the tests isolated.
This is the simplest form of tests but are still a valuable tool to create a durable and well tested application.

\subsubsection{Integration tests}
Integration tests work differently than the unit tests. 
They do not test the code itself, but the application as a whole. 
This means that these tests only go well, if no errors occur in any layer of the application. 
A top down approach is used.
The integration tests start up the \glsi{node} server, and then starts querying against it's endpoints. 
When tests are run, a test database is used. This test database is seeded with known data every time the server starts in testing mode.

\section{Continuous Integration}
A \glsi{travis} is set up to run builds, tests and more for both the front and back-end applications.

When commits are made and pushed to any one of the git repositories \glsi{travis} runs different tasks and verify the integrity of the commit.

Tasks that are run include:

\begin{enumerate}
\item \textbf{Linting}: Code style is tested to keep code quality high and consistent
\item \textbf{Build}: The source code is compiled
\item \textbf{Tests}: Unit tests, and integration tests on back-end, are run to test code functionality
\item \textbf{Code coverage}: Helps keeping track of what parts of the code is tested
\end{enumerate} 

Should any of these tasks fail, the user committing the code, will receive an email notification.
This helps both the developer who made the change, but also the developer that have to review the changes.

When a pull request is reviewed, it contains information about the test results and what the code coverage is.
If code coverage decreases in the pull request, compared to the master branch, then the developer probably did not test the code well enough. 
The lint check helps the reviewer to focus on the functional parts instead of syntactic and code style things.
