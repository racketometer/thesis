\chapter{Market analysis}
This chapter describes the findings of a market analysis for similar products. 
Here follows a short description of each product found.

\section*{SOTX Smart Badminton Racquet}
SOTX have made the product series \textit{Smart Badminton Racquet}. 
These rackets are labeled as having sensors and a microprocessor built into the handle and can communicate with \textit{Android} or \textit{iOS} smartphones via \glsi{bluetooth}  \citep{marketAnalysis:sotxsite}. 
It supports both live streaming as well as transferring data at the end of a match. 
It seems to come with a racket charger you can plug your racket into \citep{marketAnalysis:sotxhw}, but it advertises wireless charging, how those two things work together is not documented.

From the look of the app in the feature picture on an article on \textit{LinkedIn}, it seems to focus mostly on what kind of swing you perform \citep{marketAnalysis:sotxapp}, but can also keep track on how long you have been working out, calories burned and maximum speed.

\section*{Holy Pie Smart Racket}
In 2014 a \textit{Kickstarter} campaign called \textit{Holy Pie Smart Racket} launched \citep{marketAnalysis:holypie}. 
It successfully raised \$2010 with four backers. 
They made a system for both badminton and tennis. 
It is not easy to tell what this product can do since their website seems unfinished as most of the links do not work and there is still dummy text present \citep{marketAnalysis:holypi}.

\section*{USENSE}
\textit{USENSE} is a sensor that is mounted at the bottom of the racket shaft. 
There is an \textit{Android} and \textit{iOS} application so that it can connect to your smartphone through \glsi{bluetooth} 4.0+ or 2.0. 
Both specifications are mentioned \citep{marketAnalysis:usense}.
 
This one is significantly different from the other two because you do not need to buy a new badminton racket, you only buy a sensor and mount it on your own racket. 
A downside might be that it is not integrated into the racket and changes the balance of the racket.

\section*{Reflection}
All of these products seem very undocumented and their websites are either unfinished or cumbersome to understand, especially the ones written in English. 
They are only available through smartphone applications and not on desktops or web which may be useful for coaches.
