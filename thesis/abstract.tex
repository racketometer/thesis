\chapter*{Abstract}
This document describes the process of creating the software platform for the sport gadget \textit{Racket O Meter}, a multi-sensor hardware device implemented inside badminton rackets to allow for detailed measurements of movement during a badminton performance, provided by external collaborators.

The \glsi{scrum} methodology was used to manage the agile software development and to structure the work process.
By analysing expectations and use cases with user surveys and interviews, the requirements were defined and prioritized before software development commenced.

The developed software applications aimed to allow players, coaches and consultants to manage and measure badminton performances in an accessible way.

Two applications were created.
A front-end cross-platform mobile application allowing the user to communicate with the racket and persisting data on the back-end.
It further more allows consultants to create new users and more.
The applications communicate with the sensor devices with \gls{bluetooth:ble} and the back-end with the internet.

The back-end application is a web \gls{api} providing means of the mobile application to persist data and perform data analysis.
It handles user management and authentication as well.

The project results are a working cross-platform mobile application, deployed to \textit{Android}'s app store \textit{Google Play}.
It allows for consultants to create new users and players to login with auto-generated and personal passwords.
The development of the racket sensors was delayed and not ready for testing.
Because of this, a representative peripheral device was used to test \glsi{bluetooth:ble} communications.
Further more a back-end application have been developed with the afore mentioned functionality and deployed to \glsi{heroku}.
