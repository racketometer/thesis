\chapter*{Abstract}
This document describes the process of creating the software platform for the sport gadget \textit{Racket O Meter}, a multi-sensor hardware device implemented inside badminton rackets to allow for detailed measurements of movement during a badminton performance, provided by external collaborators.

The \glsi{scrum} methodology was used to manage the agile software development and to structure the work process.
By analysing expectations and use cases with user surveys and interviews, the requirements were defined and prioritized before software development commenced.

The developed software applications aimed to allow players, coaches and consultants to manage and measure badminton performances in an accessible way.

Two applications were created.
A cross-platform mobile application allows the user to communicate with the racket and persist data on the back-end.
It further more allows consultants to create new users and more.
The applications communicate with the sensor devices with \gls{bluetooth:ble} and the back-end with the internet.

The back-end application is a web \gls{api} providing means of the mobile application to persist data and perform data analysis.
It handles user management and authentication as well.

The project results are a working cross-platform mobile application, deployed to \textit{Android}'s app store \textit{Google Play}.
It allows for consultants to create new users and players to login with auto-generated and personal passwords.
The development of the racket sensors was delayed and not ready for testing.
Because of this, a representative peripheral device was used to test \glsi{bluetooth:ble} communications.
Further more a back-end application have been developed with the afore mentioned functionality and deployed to \glsi{heroku}.

\chapter*{Resumé}
Dette dokument beskriver udviklingsprocessen af softwareplatformen til en sport-gadget ved navn \textit{Racket O Meter}, der er udviklet af eksterne samarbejdspartnere.
Enheden er en multi-sensor hardware enhed der monteres på en badmintonketsjer og kan måle bevægelser under en kamp.

\Glsi{scrum} metoden har været anvendt til den agile projektudførelse og softwareudvikling.
Med analyser af brugerundersøgelser og interviews har vi defineret kravende til produktet og prioriteret de funktioner der skulle være tilgængelige.

De udviklede applikationer skal give mulighed for at spillere, trænere og konsulenter kan administrere og analysere spillerresultater og præstationer på en nem måde.

Der er udviklet to applikationer.
En cross-platform mobilapplikation der giver brugere mulighed for at kommunikere og persistere data fra ketsjeren ved hjælp af \glsi{bluetooth:ble} og internettet.
Den giver også mulighed for brugeroprettelse og andre gængse funktioner.

Serverapplikationen udstiller et web \gls{api} der giver mulighed for at persistere data om brugere og målinger, dataanalyse af målinger samt brugerautentifikation.

Resultatet af projektet er en mobilapplikation udgivet på \textit{Android}'s app-store, \textit{Google Play}.
Den giver konsulenter mulighed for at oprette nye brugere og spillere mulighed for at logge ind med deres auto-genererede eller personlige adgangskoder.
Udviklingen af ketsjersensoren blev ikke færdiggjort inden projektets afslutning, men en repræsentativ enhed, der anvender samme kommunikationsprotokol er anvendt til at verificere mobilapplikationens funktionalitet.
Yderligere er serverapplikationen blevet udgivet på \glsi{heroku}.
