\section{Project economy}
When deciding the price model of the product, the total costs of the deployed system and development tools should be compared to the consumer's expectations.

From the user survey, section \ref{sec:userSurvey}, we can see that racket prices are around kr. 1000 – 1499 and the most owned sport-gadgets was \glsi{endomondo}, free with premium subscription of kr. 15 and \glsi{gopro} with price range of kr. 1600 – 3350.

\subsection{Development}
The goal of this project was to create the products with \gls{oss} and it should preferably be free.
The project uses a single package that is a commercial product, see section \ref{sec:dependencies}.
As this package could be substituted by other packages the economic foot print of the libraries and frameworks used for the project could become zero.

\subsection{Production}
There are costs to running the applications in production.
The platform specific application stores requires developer licences that comes with a fee.
These are kr. 175 (\$25) one-time \citep{economy:play} and kr. 690 (\$99) per year \citep{economy:appStore} for \textit{Google Play} and \textit{Apple AppStore} respectively.

Several solutions are available for hosting the back-end.
As it is a \glsi{node} application any provider either serving a \glsi{node} environment or a container run-time can be used.
In table \ref{tbl:hosting} is a short overview of possible providers and their estimated prices.
All the container providers are only billing actual traffic to the running instances.
This means the actual costs of running the back-end can differ significantly from the listed prices according to the popularity of the product.

A \gls{dns} is \textit{nice to have}, for the front-end application to communicate with the back-end, without knowing the location, i.e. IP address.
The costs of this can be ignored, as the pricing is typically low and it is applicable for all providers.

\begin{table}
\caption{Price comparison for back-end hosting}
\label{tbl:hosting}

\begin{tabularx}{\textwidth}{LcL}
\textbf{Provider} & \textbf{Price per month} & \textbf{Type} \\
\hline
Amazon EC2                  & \$0.1 & Container \\
DigitalOcean                & \$10  & Container \\
Google Container Engine     & \$5   & Container \\
Heroku                      & \$7   & Container \\
Microsoft Azure             & \$14  & Container \\
mLab                        & \$15  & Only \glsi{mongodb} \\
Evennode                    & \$14  & \glsi{node} including \glsi{mongodb} \\
\hline
\end{tabularx}
\end{table}
