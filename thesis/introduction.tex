\chapter{Introduction}
This chapter gives a brief introduction to the project and how to solve it.
The main problem to solve is:

\userstory{Enable athletes to monitor, visualize and track their performances, in an accessible way, when exercising}

The overall system aims to enable collaboration between athletes, coaches, administration and other stakeholders in the badminton clubs to gain better achievements and progress for the athletes.

The main part is a ''smart'' racket that can provide the club, coaches and individual athletes with large amounts of both directly measured and deduced data about their badminton performances.
This information can help the coaches find weaknesses and show them where to focus their training for a specific player.
The second part is a management system where management staff can handle business such as economics, practice attendance and events such as parties and tournaments.
The system is illustrated in figure \ref{fig:introduction:vision}.

\graphic{1}{rich_vision.png}{The different parts of the project}{fig:introduction:vision}

The vision for this project is to revolutionize club management and progress monitoring.
The progress monitoring will be made possible with a mobile application that can gather data over \glsi{bluetooth} from sensors in the smart racket and transfer it to the management system.
The data will be available on both the mobile device and in the web application for the player, coach and club to analyse.

The club can own smart rackets for their athletes to use on occasions where the coach gathers the data for both the athlete and club to see.
The athlete should also have the option to buy its own smart racket to monitor progress at every single practice or match.
Data will bind to the athlete's individual user and will be shareable with the coaches and club.
The athletes can further more compare their progress and statistics with their friends and other athletes from the club and share training sessions and progress on social media.

\section{Relevance}
As sports wearables are on the rise with new products launching every month, most large technology companies like \textit{Apple, Inc.} and \textit{Samsung} are investing heavily into the market and \textit{Gartner, Inc.} predicts that ''\textit{From 2015 through 2017, smartwatch adoption will have 48 percent growth(...)}'' \citep{introduction:relevance:gartner}.

New products that enable users to understand their performances and activities are in high demand, but most of the wearables are centred on wristbands and smartphone applications.
These have certain limitations and the precision of the data gathered can be of poor quality.
The product of this project is different.
It is in the centre of the action and gives precise measurements of what forces act on the racket.
With this in mind the product is different in the technological sense but fits the same user demands of existing wearables.

\section{Boundary}
Since the vision is more than two man can overcome during this project's duration, boundaries are set for the specific parts that we want to achieve, figure \ref{fig:introduction:visionBoundaries}.

As mentioned in the relevance section, the part that makes this system different from a normal club management system is the progress monitoring through the mobile application and smart racket.
Since this is the unique feature, it is also going to be the focus point for the project.

\graphic{0.8}{rich_vision_with_focus.png}{The parts this project will focus on}{fig:introduction:visionBoundaries}

\section{External collaboration}
\label{sec:externalCollaboration}
As this is a software project we have allied with two external consultants.
They are responsible for the development of the physical smart racket and the electronics inside.
Further more they have great knowledge of the badminton community and a great personal network with club managers, coaches and athletes that can be used for testing, research of user demands and alike.
They will be referred to as the ''consultants'' throughout this report.
An interview with the consultants were conducted in the initial phase of the project as seen in appendix \ref{app:ch:consultantsInterview}.

\section{Work process}
The project process is managed with the agile \glsi{scrum} development framework as this is the most popular methodology in software development \citep{introduction:work:scrum}. 
In the initial weeks of the project, sprints with one and two weeks duration was tested out.
The latter fit us the most as it covered enough work to make progress and little enough to assess.
The flow of events in the sprint process is as follows.
A grooming session is held, at the end of each sprint, to discuss and define new tasks for the project, to make sure the \textit{Backlog} is prioritized and up to date.

The \textit{Backlog} is holding all the tasks to be carried out, each of which are estimated with \glspli{scrum:point}, an arbitrary value of how large a given task is.
This is an estimate and the team is ''learning by doing'' what a single \glsi{scrum:point} means.
The number of points to plan in each sprint is set, by comparing the former sprints total throughput and the expected available resources in the coming sprint.
Over the nine sprints we have completed around 120 \glspli{scrum:point} in each sprint as seen on figure \ref{fig:introduction:throughput}.

\graphic{1}{throughput}{Throughput graph showing the completed \glspli{scrum:point} of each week}{fig:introduction:throughput}

In the grooming sessions a non-formal \textit{retrospective}, i.e. reflection of the sprint, was held. 
If any tasks were not finished in the current sprint, it was decided if they should be moved into the new sprint or the \textit{Backlog} for later completion.

Visualization of tasks and their states is done with an online \gls{scrum:board}.
The tool used is \textit{Waffle}, as was decided in a pre-study, section \ref{preStudy:scrum}.
On figure \ref{fig:introduction:scrumBoard} is a screenshot of the sprint at the time of writing.
The \textit{Backlog} is in the collapsed column to the far left.
Next is the \textit{Ready} column, with tasks ready to be completed in the sprint.
Next is the \textit{In Progress} column holding all tasks currently being worked upon.
After a task is done it moves to the \textit{Review} column where it is reviewed by a team member.
When the review has approved the changes the tasks are placed in the \textit{Done} column.

\graphic{1}{scrumBoard}{Screenshot of the \textit{Waffle} scrum board}{fig:introduction:scrumBoard}

To make the management of tasks as easy as possible, the used tool synchronizes with \textit{Git} branches, i.e. in-progress work in version control. 
When a new branch is created in \textit{Git}, referencing a task number, the task it is moved to \textit{In Progress}. 
When a \textit{Pull Request}, i.e. changes ready to be merged into the \textit{master} branch, is opened, the task is moved to \textit{Review} and finally if the \textit{Pull Request} is merged and closed, the task is put in \textit{Done}.

The flow and use of this board has been nothing but fantastic.
We have had great overview of the tasks and the development as well as thesis and documentation writing have been handled with ease.

\section{Roadmap}
In planning what we can achieve during this project, a roadmap was created with overall milestones. 
The roadmap focuses on the progress of the application during the project. 
See the roadmap on figure \ref{fig:introduction:roadmap}.

When the roadmap was initially created, the ideas for how the application should look like was not set in stone.
Many of the milestones represent views in the application, but there turned out to be many views that we did not see when then roadmap was initially created.
This resulted in a slide where most of the milestones in November and December did not make it into the prototype, but instead other views did.

The roadmap was used during \gls{scrum} planning but became less of a factor the further the project got along. 
A refactored roadmap halfway through the project, when we had a better idea of what was needed, could have been useful.

\graphic{1}{Timeline_v1_2.png}{Roadmap}{fig:introduction:roadmap}

\chapter{Racket description}
This chapter will describe the available features of the racket and explain some basic terms that are relevant for the descriptions.
The chapter is mainly a short introduction to the terms based on the document in appendix \ref{app:ch:racketSensors}.

\section{Terms}
There are four terms generally used to describe badminton rackets.

\subsection*{Point of balance}
Generally this is a measurement in millimetres from the bottom of the handle and up the shaft.
Rackets are grouped into three categories according to this length.
There is no official definition of the limits of these, but the general understanding is as shown in table \ref{tab:racket:pointOfBalance}.

\begin{table}
	\begin{center}
		\begin{tabularx}{0.7\textwidth}{C|C}
			\textbf{Category} & \textbf{Length [mm]} \\
			\hline
			Head light        & < 285                \\
			Balanced          & 285-295              \\
			Head heavy        & >295                 \\
		\end{tabularx}
	\end{center}
    \caption{Categorization of a racket's point of balance}
    \label{tab:racket:pointOfBalance}
\end{table}

\subsection*{Weight}
The weight of the racket is normally in the range of 75-90 grams without strings and grip and 80-100 grams including strings and grip.
Generally the Yonex \textit{U-system} is used to group rackets as seen in table \ref{tab:racket:weight}

\begin{table}
	\begin{center}
		\begin{tabularx}{0.7\textwidth}{C|C|L}
			\textbf{Category} & \textbf{Weight [grams]} & \textbf{Description} \\
			\hline
			U1                & 95-100                  & Very heavy           \\
			U2                & 90-94                   & Heavy                \\
			U3                & 85-89                   & Normal               \\
			U4                & 80-84                   & Light                \\
			U5                & 75-79                   & Very light           \\
		\end{tabularx}
	\end{center}
    \caption{Categorization of a racket's weight}
    \label{tab:racket:weight}
\end{table}

\subsection*{Moment of inertia}
This is a measure of the racket's ability to rotate.
Different rackets require different energy amounts to rotate and typically this is in the range of 85-112 \(kg*cm^2\).
This is a key measure to calculating the features of the racket.

\subsection*{Flexibility}
As rackets are constructed of different materials they differ in their ability to bend.
This is the measure of flexibility.
According to the player's style more or less flexibility can be preferred.

\section{Sensors}
The sensors build into the racket are an accelerometer and a gyroscope measuring on all three axes as illustrated on figure \ref{fig:sensorAxes}.

Each sensor has a conversion factor to translate the measured levels into SI units.
The factors from the sensor producers are, per the consultants, not accurate to the measurements made with the racket.
Because of this, it is important to keep track of what sensors recorded the data to be able to recalculate features when new versions of the racket might ship.

\graphic{1}{sensorAxes}{Axes of measurement for the racket sensors}{fig:sensorAxes}

\section{Features}
Several features are possible to identify from the collected data.
Some requires only the data it self and others correlate to existing data, e.g. the user's own data.

\subsection*{Kinematics}
This is a feature that analyses the kinematics during a stroke.
Four measures are available from the sensors.

\begin{itemize}
    \item Angle acceleration
    \item Angle speed
    \item Linear acceleration
    \item Linear speed
\end{itemize}

Each of these can give a measure independently or be combined as a measure of the athletes technique as described below.

Theoretically all the measures combined can describe the racket's position in space.
This is however not yet possible, as the racket construction introduce some drift to the data.

\subsection*{Power}
The measure of power is really a calculation of the theoretical initial speed of the shuttlecock.

\subsection*{Technique}
The kinematic measures can be used to calculate the stroke details and can describe how ''good'', ''accurate'' or ''hard'' a given stroke was.
By correlation analysis with existing records of performances these measures can be used to rate strokes against fellow athletes or \textit{a golden standard}.

\subsection*{Choice of racket}
To help the athlete select the perfect racket, a combination of the above mentioned measures can be used to select the better racket type for the individual.
Athletes on different levels have different demands of their racket and by analyzing the kinematics and power of a stroke with different rackets, the optimum racket can be found.

\subsection*{In-game time}
This is a measure of the effective game time of the athlete, i.e. excluding breaks and time outs.

