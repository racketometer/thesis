% Tabluarx colummn fill width R and L justification
\newcolumntype{R}{>{\raggedleft\arraybackslash}X}
\newcolumntype{L}{>{\raggedright\arraybackslash}X}
\newcolumntype{C}{>{\hsize=1\hsize\centering\arraybackslash}X}
\newcolumntype{b}{X}
\newcolumntype{s}{>{\hsize=.5\hsize}X}

% ¤¤ Add graphics ¤¤ %
% Parameters
%  #1: Width ratio of full page
%  #2: Relative path from graphics folder
%  #3: Caption
%  #4: Label
%
% \graphic{1}{1.jpg}{Picture of something}{pic1}

\newcommand{\graphic}[4]{
    \begin{figure} \centering
        \includegraphics[width=#1\textwidth]{graphics/#2}
        \caption{#3}\label{#4}
    \end{figure}
}

% ¤¤ Add multipage pdf ¤¤ %
% Parameters
%  #1: Width ratio of full page
%  #2: Relative path from graphics folder
%
% \ajoutpdf{1}{1.jpg}
%
% From: http://tex.stackexchange.com/a/324786

\newcommand{\ajoutpdf}[2] {
    \pdfximage{graphics/#2}
    \multido{\i=1+1}{\the\pdflastximagepages}{
        \includegraphics[page=\i,scale=#1]{graphics/#2}
        \newpage
    }
}

% ¤¤ Add Italic to gls ¤¤ %
% Parameters
%  #1: the gls keyword
%
% \glsi{nativescript}

\newcommand{\glsi}[1]{
    \textit{\gls{#1}}
}

% ¤¤ Add Italic to Gls ¤¤ %
% Parameters
%  #1: the gls keyword
%
% \glsi{nativescript}

\newcommand{\Glsi}[1]{
    \textit{\Gls{#1}}

% ¤¤ Add Italic to glspl ¤¤ %
% Parameters
%  #1: the glspl keyword
%
% \glspli{nativescript}

\newcommand{\glspli}[1]{
    \textit{\glspl{#1}}
}

% ¤¤ Signature line ¤¤ %
% Include comment parts to show dates on line
\newcommand*{\signature}[2]{%
    \par\noindent\makebox[8cm]{\hrulefill} %\hfill\makebox[2.0in]{\hrulefill}
    \vspace{-2mm}
    \par\noindent\makebox[8cm][l]{#1\hfill#2}      %\hfill\makebox[2.0in][l]{#2}
    \vspace*{5mm}
}

% ¤¤ User stories ¤¤ %
\definecolor{usercolor}{gray}{0.9}

\newcommand{\userstory}[1]{
    \begin{tabularx}{\textwidth}{X}
        \par\noindent\textcolor{usercolor}{\hrule}
        \vspace{3mm}
        \textbf{#1}
        \vspace{-2mm}
        \par\noindent\textcolor{usercolor}{\hrule}
    \end{tabularx}
}

% ¤¤ Data field ¤¤ %
% \field[optional description]{name}{type}
\newcommand{\field}[3][]{
    \begin{tabularx}{\textwidth}{X}
        #2 \texttt{\detokenize{#3}} #1
    \end{tabularx}
}

% ¤¤ Icons ¤¤ %
\newcommand{\cmark}{\ding{51}} % cross
\newcommand{\xmark}{\ding{55}} % checkmark